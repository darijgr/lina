\documentclass[numbers=enddot,12pt,final,onecolumn,notitlepage]{scrartcl}%
\usepackage[headsepline,footsepline,manualmark]{scrlayer-scrpage}
\usepackage[all,cmtip]{xy}
\usepackage{amssymb}
\usepackage{amsmath}
\usepackage{amsthm}
\usepackage{framed}
\usepackage{comment}
\usepackage{color}
\usepackage{hyperref}
\usepackage[sc]{mathpazo}
\usepackage[T1]{fontenc}
\usepackage{needspace}
\usepackage{tabls}
%TCIDATA{OutputFilter=latex2.dll}
%TCIDATA{Version=5.50.0.2960}
%TCIDATA{LastRevised=Wednesday, September 14, 2016 19:21:29}
%TCIDATA{SuppressPackageManagement}
%TCIDATA{<META NAME="GraphicsSave" CONTENT="32">}
%TCIDATA{<META NAME="SaveForMode" CONTENT="1">}
%TCIDATA{BibliographyScheme=Manual}
%TCIDATA{Language=American English}
%BeginMSIPreambleData
\providecommand{\U}[1]{\protect\rule{.1in}{.1in}}
%EndMSIPreambleData
\newcounter{exer}
\newcounter{exera}
\theoremstyle{definition}
\newtheorem{theo}{Theorem}[section]
\newenvironment{theorem}[1][]
{\begin{theo}[#1]\begin{leftbar}}
{\end{leftbar}\end{theo}}
\newtheorem{lem}[theo]{Lemma}
\newenvironment{lemma}[1][]
{\begin{lem}[#1]\begin{leftbar}}
{\end{leftbar}\end{lem}}
\newtheorem{prop}[theo]{Proposition}
\newenvironment{proposition}[1][]
{\begin{prop}[#1]\begin{leftbar}}
{\end{leftbar}\end{prop}}
\newtheorem{defi}[theo]{Definition}
\newenvironment{definition}[1][]
{\begin{defi}[#1]\begin{leftbar}}
{\end{leftbar}\end{defi}}
\newtheorem{remk}[theo]{Remark}
\newenvironment{remark}[1][]
{\begin{remk}[#1]\begin{leftbar}}
{\end{leftbar}\end{remk}}
\newtheorem{coro}[theo]{Corollary}
\newenvironment{corollary}[1][]
{\begin{coro}[#1]\begin{leftbar}}
{\end{leftbar}\end{coro}}
\newtheorem{conv}[theo]{Convention}
\newenvironment{condition}[1][]
{\begin{conv}[#1]\begin{leftbar}}
{\end{leftbar}\end{conv}}
\newtheorem{quest}[theo]{Question}
\newenvironment{algorithm}[1][]
{\begin{quest}[#1]\begin{leftbar}}
{\end{leftbar}\end{quest}}
\newtheorem{warn}[theo]{Warning}
\newenvironment{conclusion}[1][]
{\begin{warn}[#1]\begin{leftbar}}
{\end{leftbar}\end{warn}}
\newtheorem{conj}[theo]{Conjecture}
\newenvironment{conjecture}[1][]
{\begin{conj}[#1]\begin{leftbar}}
{\end{leftbar}\end{conj}}
\newtheorem{exam}[theo]{Example}
\newenvironment{example}[1][]
{\begin{exam}[#1]\begin{leftbar}}
{\end{leftbar}\end{exam}}
\newtheorem{exmp}[exer]{Exercise}
\newenvironment{exercise}[1][]
{\begin{exmp}[#1]\begin{leftbar}}
{\end{leftbar}\end{exmp}}
\newenvironment{statement}{\begin{quote}}{\end{quote}}
\iffalse
\newenvironment{proof}[1][Proof]{\noindent\textbf{#1.} }{\ \rule{0.5em}{0.5em}}
\fi
\let\sumnonlimits\sum
\let\prodnonlimits\prod
\let\cupnonlimits\bigcup
\let\capnonlimits\bigcap
\renewcommand{\sum}{\sumnonlimits\limits}
\renewcommand{\prod}{\prodnonlimits\limits}
\renewcommand{\bigcup}{\cupnonlimits\limits}
\renewcommand{\bigcap}{\capnonlimits\limits}
\setlength\tablinesep{3pt}
\setlength\arraylinesep{3pt}
\setlength\extrarulesep{3pt}
\voffset=0cm
\hoffset=-0.7cm
\setlength\textheight{22.5cm}
\setlength\textwidth{15.5cm}
\newenvironment{verlong}{}{}
\newenvironment{vershort}{}{}
\newenvironment{noncompile}{}{}
\excludecomment{verlong}
\includecomment{vershort}
\excludecomment{noncompile}
\newcommand{\id}{\operatorname{id}}
\ihead{Notes on linear algebra}
\ohead{page \thepage}
\cfoot{}
\begin{document}

\title{Notes on linear algebra}
\author{Darij Grinberg}
\date{
%TCIMACRO{\TeXButton{today}{\today}}%
%BeginExpansion
\today
%EndExpansion
}
\maketitle
\tableofcontents

\section{Preface}

These notes are accompanying \href{http://www.math.umn.edu/~dgrinber/4242/}{a
class on applied linear algebra (Math 4242) I am giving at the University of
Minneapolis in Fall 2016} (the website of the class is
\url{http://www.math.umn.edu/~dgrinber/4242/} ). They contain both the
material of the class (although with no promise of timeliness!) and the
homework exercises (and possibly some additional exercises).

There will (probably) be no actual applications in these notes, but only the
mathematical material used in these applications. If time allows, the notes
will contain tutorials on the use of \href{http://www.sagemath.org/}{SageMath}
(a computer algebra system suited both for numerical and for algebraic computations).

Sections marked with an asterisk (*) are not a required part of the Math 4242 course.

Several good books have been written on linear algebra; these notes are not
supposed to replace any of them. Let me just mention four sources I can
recommend\footnote{I have \textbf{not} read any of the books myself (apart
from fragments). My recommendations are based on cursory skimming and random
appraisal of specific points; do not mistake my recommendations for
guarantees.}:

\begin{itemize}
\item Olver's and Shakiban's \cite{OlvSha06} is the traditional text for Math
4242 at UMN. It might be the best place to learn about the applications of
linear algebra.

\item Hefferon's \cite{Heffer16} is a free text that does things slowly but
rigorously (at least for the standards of an introductory linear-algebra
text). It has plenty of examples (and exercises with solutions), fairly
detailed proofs, and occasional applications. (Which is why it is over 500
pages long; I hope you can easily decide what to skip based on your
preferences.) Altogether, I think it does many things very well. The main
drawback is its lack of the theory of bilinear forms (but I don't know if we
will even have time for that).

\item Lankham's, Nachtergaele's and Schilling's \cite{LaNaSc16} is a set of
notes for introductory linear algebra, doing the abstract side (vector spaces,
linear maps) early on and in some detail.

\item Treil's \cite{Treil15} is another free text; this is written for a more
mathematically mature reader, and has a slight bias towards the linear algebra
useful for functional analysis.\footnote{The title of the book is a play on
Axler's \textquotedblleft Linear Algebra Done Right\textquotedblright, which
is biased towards analysis (or, rather, against algebra) to a ridiculous
extent. Axler seems to write really well, but the usefulness of this book is
severely limited by its obstinate avoidance of anything that looks too
explicit and algebraic.}
\end{itemize}

{\small [Please let the authors know if you find any errors or unclarities.
Feel free to ask me if you want your doubts resolved beforehand.]}

Also, some previous iterations of Math 4242 have left behind interesting notes:

\begin{itemize}
\item Stephen Lewis, Fall 2014, \url{http://www.stephen-lewis.net/4242/}
(enable javascript!).

\item Nathalie Sheils, Fall 2015,
\url{http://math.umn.edu/~nesheils/F15_M4242/LectureNotes.html} (yes, those
are on dropbox).
\end{itemize}

There are countless other sets of lecture notes on the internet, books in the
library, and even books on the internet if you know where to look. You can
find an overview of (published, paper) books in \cite{Drucker12} (but usually
without assessing their quality), and another (with reviews) on the MAA
website \url{http://www.maa.org/tags/linear-algebra} . (Reviews on Amazon and
goodreads are usually just good for a laugh.)

The notes you are reading are under construction, and will remain so for at
least the whole Fall term 2016. Please let me know of any errors and
unclarities you encounter (my email address is \texttt{dgrinber@umn.edu}%
)\footnote{The sourcecode of the notes is also publicly available at
\url{https://github.com/darijgr/lina} .}. Thank you!

\subsection{Acknowledgments}

I would like to thank Mark Richard for correcting a typo in the notes.

\section{\label{chp.intro}Introduction to matrices}

\subsection{Matrices and entries}

In the following, we shall study matrices filled with numbers. This is not the
most general thing to study (we could also fill matrices with other things,
such as polynomials -- and in fact, such matrices are highly useful); nor will
we be very precise about it. In fact, for the first few sections of this
chapter, we shall not even specify what we mean by \textquotedblleft
numbers\textquotedblright, even though the word \textquotedblleft
number\textquotedblright\ is far from being a well-defined notion. However, as
soon as we start caring about (say) computer calculations, we will have to
spend some words specifying our \textquotedblleft numbers\textquotedblright%
\ more precisely.

We shall use the symbol $\mathbb{N}$ for the set $\left\{  0,1,2,\ldots
\right\}  $. This is the set of all nonnegative integers.

\begin{definition}
If $n\in\mathbb{N}$ and $m\in\mathbb{N}$, then an $n\times m$\textit{-matrix}
simply means a rectangular table with $n$ rows and $m$ columns, such that each
cell is filled with a number.
\end{definition}

For example, $\left(
\begin{array}
[c]{ccc}%
1 & 7 & 2\\
-\sqrt{2} & 6 & 1/3
\end{array}
\right)  $ is a $2\times3$-matrix, whereas $\left(
\begin{array}
[c]{c}%
1\\
-2\\
0
\end{array}
\right)  $ is a $3\times1$-matrix.\footnote{For the friends of stupid examples
(me?), here are two more: $\left(
\vphantom{\begin{array}{c} a \\ a \\ a \end{array}}\right)  $ is a $3\times
0$-matrix (it contains no cells, and thus no numbers), and $\left(
\phantom{\begin{array}{ccc} a & a & a \end{array}}\right)  $ is a $0\times
3$-matrix (again, with no numbers because it has no cells). For various
technical reasons (\cite{deBoor}), it is helpful to regard such empty matrices
as different.}

\begin{definition}
The word \textquotedblleft\textit{matrix}\textquotedblright\ will encompass
all $n\times m$-matrices for all possible values of $n$ and $m$.
\end{definition}

\begin{definition}
The \textit{dimensions} of an $n\times m$-matrix $A$ are the two integers $n$
and $m$. When they are equal (that is, $n=m$), we say that $A$ is a
\textit{square matrix}, and call $n$ its \textit{size}.
\end{definition}

\begin{definition}
If $A$ is an $n\times m$-matrix, and if $i\in\left\{  1,2,\ldots,n\right\}  $
and $j\in\left\{  1,2,\ldots,m\right\}  $, then $A_{i,j}$ will denote the
entry of $A$ in row $i$ and column $j$. This entry is also called the $\left(
i,j\right)  $\textit{-th entry of }$A$ (or simply the $\left(  i,j\right)
$\textit{-entry of }$A$).
\end{definition}

For example, $\left(
\begin{array}
[c]{ccc}%
1 & 7 & 2\\
-\sqrt{2} & 6 & 1/3
\end{array}
\right)  _{1,3}=2$. Note that this notation $A_{i,j}$ (for the $\left(
i,j\right)  $-th entry of a matrix $A$) is not standard in literature! Some
authors instead use $A_{j}^{i}$ (the $i$ on top is not an exponent, but just a
superscript) or $a_{i,j}$ (where $a$ is the lowercase letter corresponding to
the uppercase letter $A$ denoting the matrix\footnote{This notation is bad for
two reasons: First, it forces you to always denote matrices by uppercase
letters; second, it doesn't let you write things like $\left(
\begin{array}
[c]{ccc}%
1 & 7 & 2\\
-\sqrt{2} & 6 & 1/3
\end{array}
\right)  _{1,3}$.}). Many authors often drop the comma between $i$ and $j$ (so
they call it $A_{ij}$ or $a_{ij}$); this notation is slightly ambiguous (does
$A_{132}$ mean $A_{13,2}$ or $A_{1,23}$ ?). Unfortunately, some authors use
the notation $A_{i,j}$ for something else called a \textit{cofactor} of $A$
(which is, in a sense, quite the opposite of the $\left(  i,j\right)  $-th
entry of $A$); but we will never do this here (and probably we will not really
get into cofactors anyway).

\subsection{The matrix builder notation}

I would like to do something interesting, but I am forced to introduce more
notations. Please have patience with me. Let me introduce a notation for
building a matrix out of a bunch of entries:

\begin{definition}
Let $n\in\mathbb{N}$. Assume that you are given a number $a_{i,j}$ for each
pair $\left(  i,j\right)  $ of an integer $i\in\left\{  1,2,\ldots,n\right\}
$ and $j\in\left\{  1,2,\ldots,m\right\}  $. Then, $\left(  a_{i,j}\right)
_{1\leq i\leq n,\ 1\leq j\leq m}$ shall denote the $n\times m$-matrix whose
$\left(  i,j\right)  $-th entry is $a_{i,j}$ for all $i\in\left\{
1,2,\ldots,n\right\}  $ and $j\in\left\{  1,2,\ldots,m\right\}  $. (To say it
differently: $\left(  a_{i,j}\right)  _{1\leq i\leq n,\ 1\leq j\leq m}$ shall
denote the $n\times m$-matrix $A$ such that $A_{i,j}=a_{i,j}$ for all
$i\in\left\{  1,2,\ldots,n\right\}  $ and $j\in\left\{  1,2,\ldots,m\right\}
$.)
\end{definition}

Some examples:

\begin{itemize}
\item We have $\left(  i-j\right)  _{1\leq i\leq2,\ 1\leq j\leq3}=\left(
\begin{array}
[c]{ccc}%
1-1 & 1-2 & 1-3\\
2-1 & 2-2 & 2-3
\end{array}
\right)  =\left(
\begin{array}
[c]{ccc}%
0 & -1 & -2\\
1 & 0 & -1
\end{array}
\right)  $.

\item We have $\left(  i+j\right)  _{1\leq i\leq3,\ 1\leq j\leq2}=\left(
\begin{array}
[c]{cc}%
1+1 & 1+2\\
2+1 & 2+2\\
3+1 & 3+2
\end{array}
\right)  =\left(
\begin{array}
[c]{cc}%
2 & 3\\
3 & 4\\
4 & 5
\end{array}
\right)  $.

\item We have $\left(  \dfrac{i+1}{j}\right)  _{1\leq i\leq3,\ 1\leq j\leq
3}=\left(
\begin{array}
[c]{ccc}%
\dfrac{1+1}{1} & \dfrac{1+1}{2} & \dfrac{1+1}{3}\\
\dfrac{2+1}{1} & \dfrac{2+1}{2} & \dfrac{2+1}{3}\\
\dfrac{3+1}{1} & \dfrac{3+1}{2} & \dfrac{3+1}{3}%
\end{array}
\right)  =\left(
\begin{array}
[c]{ccc}%
2 & 1 & \dfrac{2}{3}\\
3 & \dfrac{3}{2} & 1\\
4 & 2 & \dfrac{4}{3}%
\end{array}
\right)  $.

\item We have $\left(  \dfrac{i-j}{i+j}\right)  _{1\leq i\leq3,\ 1\leq j\leq
2}=\left(
\begin{array}
[c]{cc}%
\dfrac{1-1}{1+1} & \dfrac{1-2}{1+2}\\
\dfrac{2-1}{2+1} & \dfrac{2-2}{2+2}\\
\dfrac{3-1}{3+1} & \dfrac{3-2}{3+2}%
\end{array}
\right)  =\left(
\begin{array}
[c]{cc}%
0 & -\dfrac{1}{3}\\
\dfrac{1}{3} & 0\\
\dfrac{1}{2} & \dfrac{1}{5}%
\end{array}
\right)  $.
\end{itemize}

The notation $\left(  a_{i,j}\right)  _{1\leq i\leq n,\ 1\leq j\leq m}$ is
fairly standard (you will be understood), though again there are variations in
the literature.

We used the two letters $i$ and $j$ in the notation $\left(  a_{i,j}\right)
_{1\leq i\leq n,\ 1\leq j\leq m}$, but we could just as well have picked any
other two letters (as long as they aren't already taken for something else).
For example, $\left(  xy\right)  _{1\leq x\leq2,\ 1\leq y\leq2}=\left(
\begin{array}
[c]{cc}%
1\cdot1 & 1\cdot2\\
2\cdot1 & 2\cdot2
\end{array}
\right)  =\left(
\begin{array}
[c]{cc}%
1 & 2\\
2 & 4
\end{array}
\right)  $.

Of course, if you decompose an $n\times m$-matrix $A$ into its entries, and
then assemble these entries back into an $n\times m$-matrix (arranged in the
same way as in $A$), then you get back $A$. In other words: For every $n\times
m$-matrix $A$, we have%
\[
\left(  A_{i,j}\right)  _{1\leq i\leq n,\ 1\leq j\leq m}=A.
\]


\subsection{Row and column vectors}

Here is some more terminology:

\begin{definition}
Let $n\in\mathbb{N}$. A \textit{row vector of size }$n$ means a $1\times
n$-matrix. A \textit{column vector of size }$n$ means an $n\times1$-matrix.
\end{definition}

For example, $\left(
\begin{array}
[c]{cc}%
a & b
\end{array}
\right)  $ is a row vector of size $2$, while $\left(
\begin{array}
[c]{c}%
a\\
b
\end{array}
\right)  $ is a column vector of size $2$.

\subsection{Transposes}

\begin{definition}
The transpose of an $n\times m$-matrix $A$ is defined to be the $m\times
n$-matrix $\left(  A_{j,i}\right)  _{1\leq i\leq m,\ 1\leq j\leq n}$. It is
denoted by $A^{T}$.
\end{definition}

Let us unravel this confusing-looking definition! It says that the transpose
of an $n\times m$-matrix $A$ is the $m\times n$-matrix whose $\left(
i,j\right)  $-th entry (for $i\in\left\{  1,2,\ldots,m\right\}  $ and
$j\in\left\{  1,2,\ldots,n\right\}  $) is the $\left(  j,i\right)  $-th entry
of $A$. So the transpose of $A$ has the very same entries as $A$, but in
different position: namely, the entry in position $\left(  i,j\right)  $ gets
moved into position $\left(  j,i\right)  $. In other words, the entry that was
in row $i$ and column $j$ gets moved into column $i$ and row $j$. So, visually
speaking, the transpose of the matrix $A$ is obtained by \textquotedblleft
reflecting $A$ around the diagonal\textquotedblright. Some examples should
help clarify this:%
\begin{align*}
\left(
\begin{array}
[c]{ccc}%
a & b & c\\
a^{\prime} & b^{\prime} & c^{\prime}%
\end{array}
\right)  ^{T}  &  =\left(
\begin{array}
[c]{cc}%
a & a^{\prime}\\
b & b^{\prime}\\
c & c^{\prime}%
\end{array}
\right)  ;\\
\left(
\begin{array}
[c]{cc}%
a & b\\
a^{\prime} & b^{\prime}%
\end{array}
\right)  ^{T}  &  =\left(
\begin{array}
[c]{cc}%
a & a^{\prime}\\
b & b^{\prime}%
\end{array}
\right)  ;\\
\left(
\begin{array}
[c]{c}%
a\\
b\\
c
\end{array}
\right)  ^{T}  &  =\left(
\begin{array}
[c]{ccc}%
a & b & c
\end{array}
\right)  .
\end{align*}


Transposes have many uses, but for now we stress one particular use: as a
space-saving device. Namely, if you work with column vectors, you quickly
notice that they take up a lot of vertical space in writing: just see by how
much the column vector $\left(
\begin{array}
[c]{c}%
4\\
-1\\
2\\
0
\end{array}
\right)  $ has stretched the spacing between its line and the lines above and
below\footnote{Additionally, column vectors of size $2$ have the annoying
property that they can get confused for binomial coefficients. To wit,
$\left(
\begin{array}
[c]{c}%
4\\
2
\end{array}
\right)  $ denotes a column vector, whereas $\dbinom{4}{2}$ denotes a binomial
coefficient (which equals the number $6$). The only way to tell them apart is
by the amount of empty space between the parentheses and the entries; this is
not a very reliable way to keep different notations apart.}! It is much more
economical to rewrite it as the transpose of a row vector: $\left(
\begin{array}
[c]{cccc}%
4 & -1 & 2 & 0
\end{array}
\right)  ^{T}$. It is furthermore common to write row vectors as tuples (i.e.,
put commas between their entries instead of leaving empty space); thus, the
row vector $\left(
\begin{array}
[c]{cccc}%
4 & -1 & 2 & 0
\end{array}
\right)  $ becomes $\left(  4,-1,2,0\right)  $ (which takes up less space),
and our column vector above becomes $\left(  4,-1,2,0\right)  ^{T}$.

The transpose of a matrix $A$ is also denoted by $A^{t}$ or $^{T}A$ or $^{t}A$
by various authors (not me).

\subsection{Addition, scaling and multiplication}

Matrices can (sometimes) be added, (always) be scaled and (sometimes) be
multiplied. Let me explain:

\begin{definition}
Let $A$ and $B$ be two matrices of the same dimensions (that is, they have the
same number of rows, and the same number of columns). Then, $A+B$ denotes the
matrix obtained by adding each entry of $A$ to the corresponding entry of $B$.
Or, to write it more formally: If $A$ and $B$ are two $n\times m$-matrices,
then%
\[
A+B=\left(  A_{i,j}+B_{i,j}\right)  _{1\leq i\leq n,\ 1\leq j\leq m}.
\]

\end{definition}

For example, $\left(
\begin{array}
[c]{ccc}%
a & b & c\\
d & e & f
\end{array}
\right)  +\left(
\begin{array}
[c]{ccc}%
a^{\prime} & b^{\prime} & c^{\prime}\\
d^{\prime} & e^{\prime} & f^{\prime}%
\end{array}
\right)  =\left(
\begin{array}
[c]{ccc}%
a+a^{\prime} & b+b^{\prime} & c+c^{\prime}\\
d+d^{\prime} & e+e^{\prime} & f+f^{\prime}%
\end{array}
\right)  $. (I am increasingly using variables instead of actual numbers,
because they make it easier to see what entry is going where.) On the other
hand, the two matrices $\left(
\begin{array}
[c]{c}%
a\\
b
\end{array}
\right)  $ and $\left(
\begin{array}
[c]{cc}%
c & d
\end{array}
\right)  $ cannot be added (since they have different dimensions\footnote{The
dimensions of the former matrix are $2$ and $1$, whereas the dimensions of the
latter matrix are $1$ and $2$. Even though they are equal \textbf{up to
order}, they do not count as equal.}).

So we now know hot add two matrices.

\begin{definition}
Let $A$ be a matrix, and $\lambda$ be a number. Then, $\lambda A$ (or
$\lambda\cdot A$) denotes the matrix obtained by multiplying each entry of $A$
by $\lambda$. In other words: If $A$ is an $n\times m$-matrix, then%
\[
\lambda A=\left(  \lambda A_{i,j}\right)  _{1\leq i\leq n,\ 1\leq j\leq m}.
\]


We write $-A$ for $\left(  -1\right)  A$.
\end{definition}

For example, $\lambda\left(
\begin{array}
[c]{ccc}%
a & b & c\\
d & e & f
\end{array}
\right)  =\left(
\begin{array}
[c]{ccc}%
\lambda a & \lambda b & \lambda c\\
\lambda d & \lambda e & \lambda f
\end{array}
\right)  $.

So now we know how to scale a matrix. (\textquotedblleft To
scale\textquotedblright\ means to multiply by a number.)

\textquotedblleft Scaling\textquotedblright\ is often called \textquotedblleft
scalar multiplication\textquotedblright\ (but this is confusing terminology,
since \textquotedblleft scalar product\textquotedblright\ means something
completely different).

With scaling and addition defined, we obtain subtraction for free:

\begin{definition}
Let $A$ and $B$ be two matrices of the same dimensions. Then, $A-B$ denotes
the matrix $A+\left(  -B\right)  =A+\left(  -1\right)  B$.
\end{definition}

For example, $\left(
\begin{array}
[c]{ccc}%
a & b & c\\
d & e & f
\end{array}
\right)  -\left(
\begin{array}
[c]{ccc}%
a^{\prime} & b^{\prime} & c^{\prime}\\
d^{\prime} & e^{\prime} & f^{\prime}%
\end{array}
\right)  =\left(
\begin{array}
[c]{ccc}%
a-a^{\prime} & b-b^{\prime} & c-c^{\prime}\\
d-d^{\prime} & e-e^{\prime} & f-f^{\prime}%
\end{array}
\right)  $.

Now, to the more interesting part: multiplying matrices. This is \textbf{not}
done by multiplying corresponding entries! (Why not? Well, it wouldn't make
for a particularly useful notion.) Instead, the definition goes as follows:

\begin{definition}
\label{def.AB}Let $n\in\mathbb{N}$, $m\in\mathbb{N}$ and $m\in\mathbb{N}$. Let
$A$ be an $n\times m$-matrix. Let $B$ be an $m\times p$-matrix. (Thus, $A$ has
to have $m$ columns, while $B$ has to have $m$ rows; other than this, the two
matrices do not need to have any relation to each other.) The product $AB$ of
these two matrices is defined as follows:%
\[
AB=\left(  \underbrace{A_{i,1}B_{1,j}+A_{i,2}B_{2,j}+\cdots+A_{i,m}B_{m,j}%
}_{\substack{\text{This is the sum of the }m\text{ terms of the form}%
\\A_{i,k}B_{k,j}\text{, for }k\text{ ranging over }\left\{  1,2,\ldots
,m\right\}  }}\right)  _{1\leq i\leq n,\ 1\leq j\leq p}.
\]
This is an $n\times p$-matrix.
\end{definition}

This definition is somewhat overwhelming, so let me rewrite it in words and
give some examples:

It says that the product $AB$ is well-defined whenever $A$ has as many columns
as $B$ has rows. In this case, $AB$ is the $n\times p$-matrix whose $\left(
i,j\right)  $-th entry is obtained by adding together:

\begin{itemize}
\item the product $A_{i,1}B_{1,j}$ of the $\left(  i,1\right)  $-th entry of
$A$ with the $\left(  1,j\right)  $-th entry of $B$;

\item the product $A_{i,2}B_{2,j}$ of the $\left(  i,2\right)  $-th entry of
$A$ with the $\left(  2,j\right)  $-th entry of $B$;

\item and so on;

\item the product $A_{i,m}B_{m,j}$ of the $\left(  i,m\right)  $-th entry of
$A$ with the $\left(  m,j\right)  $-th entry of $B$.
\end{itemize}

In other words, $AB$ is the matrix whose $\left(  i,j\right)  $-th entry is
obtained by multiplying each entry of the $i$-th row of $A$ with the
corresponding entry of the $j$-th column of $B$, and then adding together all
these products. The word \textquotedblleft corresponding\textquotedblright%
\ means that the $1$-st entry of the $i$-th row of $A$ gets multiplied with
the $1$-st entry of the $j$-th column of $B$, the $2$-nd entry with the $2$-nd
entry, etc.. In particular, for this to make sense, the $i$-th row of $A$ and
the $j$-th column of $B$ have to have the same number of entries. This is why
we required that $A$ has as many columns as $B$ has rows!

I promised examples. Here are four:%
\begin{align*}
\left(
\begin{array}
[c]{cc}%
a & b\\
a^{\prime} & b^{\prime}%
\end{array}
\right)  \left(
\begin{array}
[c]{cc}%
x & x^{\prime}\\
y & y^{\prime}%
\end{array}
\right)   &  =\left(
\begin{array}
[c]{cc}%
ax+by & ax^{\prime}+by^{\prime}\\
a^{\prime}x+b^{\prime}y & a^{\prime}x^{\prime}+b^{\prime}y^{\prime}%
\end{array}
\right)  ;\\
\left(
\begin{array}
[c]{ccc}%
a & b & c\\
a^{\prime} & b^{\prime} & c^{\prime}%
\end{array}
\right)  \left(
\begin{array}
[c]{c}%
x\\
y\\
z
\end{array}
\right)   &  =\left(
\begin{array}
[c]{c}%
ax+by+cz\\
a^{\prime}x+b^{\prime}y+c^{\prime}z
\end{array}
\right)  ;\\
\left(
\begin{array}
[c]{cc}%
a & b
\end{array}
\right)  \left(
\begin{array}
[c]{c}%
x\\
y
\end{array}
\right)   &  =\left(
\begin{array}
[c]{c}%
ax+by
\end{array}
\right)  ;\\
\left(
\begin{array}
[c]{c}%
a\\
b
\end{array}
\right)  \left(
\begin{array}
[c]{cc}%
x & y
\end{array}
\right)   &  =\left(
\begin{array}
[c]{cc}%
ax & ay\\
bx & by
\end{array}
\right)
\end{align*}
(note how in the fourth example, we don't see any plus signs, because each of
the sums has only one addend).

We can also denote the product $AB$ by $A\cdot B$ (though few people ever do this).

We have thus learnt how to multiply matrices. Notice that the $\left(
i,j\right)  $-th entry of the product $AB$ depends only on the $i$-th row of
$A$ and the $j$-th column of $B$. Why did we pick this strange definition,
rather than something simpler, like multiplying entry by entry, or at least
row by row? Well, \textquotedblleft entry by entry\textquotedblright\ is too
simple (you will see later what matrix multiplication is good for;
\textquotedblleft entry by entry\textquotedblright\ is useless in comparison),
whereas \textquotedblleft row by row\textquotedblright\ would be lacking many
of the nice properties that we will see later (e.g., our matrix multiplication
satisfies the associativity law $\left(  AB\right)  C=A\left(  BC\right)  $,
while \textquotedblleft row by row\textquotedblright\ does not).

\subsection{The matrix product rewritten}

Let me show another way to restate our above definition of a product of two
matrices. First, one more notation:

\begin{definition}
Let $A$ be an $n\times m$-matrix.

\textbf{(a)} If $i\in\left\{  1,2,\ldots,n\right\}  $, then
$\operatorname*{row}\nolimits_{i}A$ will denote the $i$-th row of $A$. This is
a row vector of size $m$ (that is, a $1\times m$-matrix), and is formally
defined as
\[
\left(  A_{i,y}\right)  _{1\leq x\leq1,\ 1\leq y\leq m}=\left(
\begin{array}
[c]{cccc}%
A_{i,1} & A_{i,2} & \cdots & A_{i,m}%
\end{array}
\right)
\]
(notice how $i$ is kept fixed but $y$ is ranging from $1$ to $m$ here).

\textbf{(b)} If $j\in\left\{  1,2,\ldots,m\right\}  $, then
$\operatorname*{col}\nolimits_{j}A$ will denote the $j$-th column of $A$. This
is a column vector of size $n$ (that is, an $n\times1$-matrix), and is
formally defined as
\[
\left(  A_{x,j}\right)  _{1\leq x\leq n,\ 1\leq y\leq1}=\left(
\begin{array}
[c]{c}%
A_{1,j}\\
A_{2,j}\\
\vdots\\
A_{n,j}%
\end{array}
\right)  .
\]

\end{definition}

\begin{example}
If $A=\left(
\begin{array}
[c]{ccc}%
a & b & c\\
d & e & f
\end{array}
\right)  $, then $\operatorname*{row}\nolimits_{2}A=\left(
\begin{array}
[c]{ccc}%
d & e & f
\end{array}
\right)  $ and $\operatorname*{col}\nolimits_{2}A=\left(
\begin{array}
[c]{c}%
b\\
e
\end{array}
\right)  $.
\end{example}

Now, we observe that if $R$ is a row vector of some size $m$, and if $C$ is a
column vector of size $m$, then $RC$ is a $1\times1$-matrix. More precisely:
The product of a row vector $\left(
\begin{array}
[c]{cccc}%
r_{1} & r_{2} & \cdots & r_{m}%
\end{array}
\right)  $ and a column vector $\left(
\begin{array}
[c]{c}%
c_{1}\\
c_{2}\\
\vdots\\
c_{m}%
\end{array}
\right)  $ is given by
\begin{equation}
\left(
\begin{array}
[c]{cccc}%
r_{1} & r_{2} & \cdots & r_{m}%
\end{array}
\right)  \left(
\begin{array}
[c]{c}%
c_{1}\\
c_{2}\\
\vdots\\
c_{m}%
\end{array}
\right)  =\left(
\begin{array}
[c]{c}%
r_{1}c_{1}+r_{2}c_{2}+\cdots+r_{m}c_{m}%
\end{array}
\right)  . \label{eq.matrix-prod.RC}%
\end{equation}
We shall often equate a $1\times1$-matrix with its (unique) entry; so the
equality (\ref{eq.matrix-prod.RC}) rewrites as%
\begin{equation}
\left(
\begin{array}
[c]{cccc}%
r_{1} & r_{2} & \cdots & r_{m}%
\end{array}
\right)  \left(
\begin{array}
[c]{c}%
c_{1}\\
c_{2}\\
\vdots\\
c_{m}%
\end{array}
\right)  =r_{1}c_{1}+r_{2}c_{2}+\cdots+r_{m}c_{m}. \label{eq.matrix-prod.RC.1}%
\end{equation}


Now, we can see the following:

\begin{proposition}
\label{prop.matrix-prod.rc}Let $n\in\mathbb{N}$, $m\in\mathbb{N}$ and
$m\in\mathbb{N}$. Let $A$ be an $n\times m$-matrix. Let $B$ be an $m\times
p$-matrix. Then, for every $i\in\left\{  1,2,\ldots,n\right\}  $ and
$j\in\left\{  1,2,\ldots,p\right\}  $, the $\left(  i,j\right)  $-th entry of
$AB$ equals the product of the $i$-th row of $A$ and the $j$-th column of $B$.
In formulas:%
\begin{equation}
\left(  AB\right)  _{i,j}=\operatorname*{row}\nolimits_{i}A\cdot
\operatorname*{col}\nolimits_{j}B \label{eq.prop.matrix-prod.rc.1}%
\end{equation}
for every $i\in\left\{  1,2,\ldots,n\right\}  $ and $j\in\left\{
1,2,\ldots,p\right\}  $. Thus,
\begin{align*}
AB  &  =\left(  \operatorname*{row}\nolimits_{i}A\cdot\operatorname*{col}%
\nolimits_{j}B\right)  _{1\leq i\leq n,\ 1\leq j\leq p}\\
&  =\left(
\begin{array}
[c]{cccc}%
\operatorname*{row}\nolimits_{1}A\cdot\operatorname*{col}\nolimits_{1}B &
\operatorname*{row}\nolimits_{1}A\cdot\operatorname*{col}\nolimits_{2}B &
\cdots & \operatorname*{row}\nolimits_{1}A\cdot\operatorname*{col}%
\nolimits_{p}B\\
\operatorname*{row}\nolimits_{2}A\cdot\operatorname*{col}\nolimits_{1}B &
\operatorname*{row}\nolimits_{2}A\cdot\operatorname*{col}\nolimits_{2}B &
\cdots & \operatorname*{row}\nolimits_{2}A\cdot\operatorname*{col}%
\nolimits_{p}B\\
\vdots & \vdots & \ddots & \vdots\\
\operatorname*{row}\nolimits_{n}A\cdot\operatorname*{col}\nolimits_{1}B &
\operatorname*{row}\nolimits_{n}A\cdot\operatorname*{col}\nolimits_{2}B &
\cdots & \operatorname*{row}\nolimits_{n}A\cdot\operatorname*{col}%
\nolimits_{p}B
\end{array}
\right)  .
\end{align*}

\end{proposition}

\begin{proof}
[Proof of Proposition \ref{prop.matrix-prod.rc}.]By the definition of $AB$, we
have%
\[
AB=\left(  A_{i,1}B_{1,j}+A_{i,2}B_{2,j}+\cdots+A_{i,m}B_{m,j}\right)  _{1\leq
i\leq n,\ 1\leq j\leq p}.
\]
In other words,%
\begin{equation}
\left(  AB\right)  _{i,j}=A_{i,1}B_{1,j}+A_{i,2}B_{2,j}+\cdots+A_{i,m}B_{m,j}
\label{pf.prop.matrix-prod.rc.1}%
\end{equation}
for every $i\in\left\{  1,2,\ldots,n\right\}  $ and $j\in\left\{
1,2,\ldots,p\right\}  $.

Now, let us prove Proposition \ref{prop.matrix-prod.rc}. It is clearly enough
to prove (\ref{eq.prop.matrix-prod.rc.1}). So let's do this. Let $i\in\left\{
1,2,\ldots,n\right\}  $ and $j\in\left\{  1,2,\ldots,p\right\}  $. Then,%
\begin{align*}
\operatorname*{row}\nolimits_{i}A  &  =\left(
\begin{array}
[c]{cccc}%
A_{i,1} & A_{i,2} & \cdots & A_{i,m}%
\end{array}
\right)  \ \ \ \ \ \ \ \ \ \ \text{and}\\
\operatorname*{col}\nolimits_{j}B  &  =\left(
\begin{array}
[c]{c}%
B_{1,j}\\
B_{2,j}\\
\vdots\\
B_{m,j}%
\end{array}
\right)  .
\end{align*}
Hence,%
\begin{align*}
\operatorname*{row}\nolimits_{i}A\cdot\operatorname*{col}\nolimits_{j}B  &
=\left(
\begin{array}
[c]{cccc}%
A_{i,1} & A_{i,2} & \cdots & A_{i,m}%
\end{array}
\right)  \left(
\begin{array}
[c]{c}%
B_{1,j}\\
B_{2,j}\\
\vdots\\
B_{m,j}%
\end{array}
\right) \\
&  =A_{i,1}B_{1,j}+A_{i,2}B_{2,j}+\cdots+A_{i,m}B_{m,j}.
\end{align*}
Comparing this with (\ref{pf.prop.matrix-prod.rc.1}), we obtain $\left(
AB\right)  _{i,j}=\operatorname*{row}\nolimits_{i}A\cdot\operatorname*{col}%
\nolimits_{j}B$. Thus, we have proven (\ref{eq.prop.matrix-prod.rc.1}).
\end{proof}

\subsection{Properties of matrix operations}

The operations of adding, scaling and multiplying matrices, in many aspects,
\textquotedblleft behave almost as nicely as numbers\textquotedblright.
Specifically, I mean that they satisfy a bunch of laws that numbers satisfy:

\begin{proposition}
\label{prop.matrix-laws.1}Let $n\in\mathbb{N}$ and $m\in\mathbb{N}$. Then:

\textbf{(a)} We have $A+B=B+A$ for any two $n\times m$-matrices $A$ and $B$.
(This is called \textquotedblleft commutativity of addition\textquotedblright.)

\textbf{(b)} We have $A+\left(  B+C\right)  =\left(  A+B\right)  +C$ for any
three $n\times m$-matrices $A$, $B$ and $C$. (This is called \textquotedblleft
associativity of addition\textquotedblright.)

\textbf{(c)} We have $\lambda\left(  A+B\right)  =\lambda A+\lambda B$ for any
number $\lambda$ and any two $n\times m$-matrices $A$ and $B$.

Let furthermore $p\in\mathbb{N}$. Then:

\textbf{(d)} We have $A\left(  B+C\right)  =AB+AC$ for any $n\times m$-matrix
$A$ and any two $m\times p$-matrices $B$ and $C$. (This is called
\textquotedblleft left distributivity\textquotedblright.)

\textbf{(e)} We have $\left(  A+B\right)  C=AC+BC$ for any two $n\times
m$-matrices $A$ and $B$ and any $m\times p$-matrix $C$. (This is called
\textquotedblleft right distributivity\textquotedblright.)

\textbf{(f)} We have $\lambda\left(  AB\right)  =\left(  \lambda A\right)
B=A\left(  \lambda B\right)  $ for any number $\lambda$, any $n\times
m$-matrix $A$ and any $m\times p$-matrix $B$.

Finally, let $q\in\mathbb{N}$. Then:

\textbf{(g)} We have $A\left(  BC\right)  =\left(  AB\right)  C$ for any
$n\times m$-matrix $A$, any $m\times p$-matrix $B$ and any $p\times q$-matrix
$C$. (This is called \textquotedblleft associativity of
multiplication\textquotedblright.)
\end{proposition}

\begin{example}
Most parts of Proposition \ref{prop.matrix-laws.1} are fairly easy to
visualize and to prove. Let me give an example for the least obvious one: part
\textbf{(g)}.

Part \textbf{(g)} essentially says that $A\left(  BC\right)  =\left(
AB\right)  C$ holds for any three matrices $A$, $B$ and $C$ for which the
products $AB$ and $BC$ are well-defined (i.e., $A$ has as many columns as $B$
has rows, and $B$ has as many columns as $C$ has rows). For example, take
$n=1$, $m=3$, $p=2$ and $q=3$. Set%
\[
A=\left(
\begin{array}
[c]{ccc}%
a & b & c
\end{array}
\right)  ,\ \ \ \ \ \ \ \ \ \ B=\left(
\begin{array}
[c]{cc}%
d & d^{\prime}\\
e & e^{\prime}\\
f & f^{\prime}%
\end{array}
\right)  ,\ \ \ \ \ \ \ \ \ \ C=\left(
\begin{array}
[c]{ccc}%
x & y & z\\
x^{\prime} & y^{\prime} & z^{\prime}%
\end{array}
\right)  .
\]
Then,%
\[
AB=\left(
\begin{array}
[c]{cc}%
ad+be+cf & ad^{\prime}+be^{\prime}+cf^{\prime}%
\end{array}
\right)
\]
and thus%
\begin{align*}
\left(  AB\right)  C  &  =\left(
\begin{array}
[c]{cc}%
ad+be+cf & ad^{\prime}+be^{\prime}+cf^{\prime}%
\end{array}
\right)  \left(
\begin{array}
[c]{ccc}%
x & y & z\\
x^{\prime} & y^{\prime} & z^{\prime}%
\end{array}
\right) \\
&  =\left(
\begin{array}
[c]{c}%
ad^{\prime}x^{\prime}+be^{\prime}x^{\prime}+cf^{\prime}x^{\prime
}+bxe+adx+cfx\\
ad^{\prime}y^{\prime}+be^{\prime}y^{\prime}+cf^{\prime}y^{\prime
}+bye+ady+cfy\\
ad^{\prime}z^{\prime}+be^{\prime}z^{\prime}+cf^{\prime}z^{\prime}+bze+adz+cfz
\end{array}
\right)  ^{T}%
\end{align*}
after some computation. (Here, we have written the result as a transpose of a
column vector, because if we had written it as a row vector, it would not fit
on this page.) But
\[
BC=\left(
\begin{array}
[c]{cc}%
d & d^{\prime}\\
e & e^{\prime}\\
f & f^{\prime}%
\end{array}
\right)  \left(
\begin{array}
[c]{ccc}%
x & y & z\\
x^{\prime} & y^{\prime} & z^{\prime}%
\end{array}
\right)  =\left(
\begin{array}
[c]{ccc}%
d^{\prime}x^{\prime}+dx & d^{\prime}y^{\prime}+dy & d^{\prime}z^{\prime}+dz\\
e^{\prime}x^{\prime}+xe & e^{\prime}y^{\prime}+ye & e^{\prime}z^{\prime}+ze\\
f^{\prime}x^{\prime}+fx & f^{\prime}y^{\prime}+fy & f^{\prime}z^{\prime}+fz
\end{array}
\right)
\]
and as before
\[
A\left(  BC\right)  =\left(
\begin{array}
[c]{c}%
ad^{\prime}x^{\prime}+be^{\prime}x^{\prime}+cf^{\prime}x^{\prime
}+bxe+adx+cfx\\
ad^{\prime}y^{\prime}+be^{\prime}y^{\prime}+cf^{\prime}y^{\prime
}+bye+ady+cfy\\
ad^{\prime}z^{\prime}+be^{\prime}z^{\prime}+cf^{\prime}z^{\prime}+bze+adz+cfz
\end{array}
\right)  ^{T}.
\]
Hence, $\left(  AB\right)  C=A\left(  BC\right)  $. Thus, our example confirms
Proposition \ref{prop.matrix-laws.1} \textbf{(g)}.
\end{example}

The laws of Proposition \ref{prop.matrix-laws.1} allow you to do many formal
manipulations with matrices similarly to how you are used to work with
numbers. For example, if you have $n$ matrices $A_{1},A_{2},\ldots,A_{n}$ such
that successive matrices can be multiplied (i.e., for each $i\in\left\{
1,2,\ldots,n-1\right\}  $, the matrix $A_{i}$ has as many columns as $A_{i+1}$
has rows), then the product $A_{1}A_{2}\cdots A_{n}$ is well-defined: you can
parenthesize it in any order, and the result will always be the same. For
example, the product $ABCD$ of four matrices $A,B,C,D$ can be computed in any
of the five ways%
\[
\left(  \left(  AB\right)  C\right)  D,\ \ \ \ \ \ \ \ \ \ \left(  AB\right)
\left(  CD\right)  ,\ \ \ \ \ \ \ \ \ \ \left(  A\left(  BC\right)  \right)
D,\ \ \ \ \ \ \ \ \ \ A\left(  \left(  BC\right)  D\right)
,\ \ \ \ \ \ \ \ \ \ A\left(  B\left(  CD\right)  \right)  ,
\]
and all of them lead to the same result. This is called \textit{general
associativity} and is not obvious (even if you know that Proposition
\ref{prop.matrix-laws.1} \textbf{(g)} holds).

Please take a moment to appreciate general associativity! Without it, we could
not make sense of products like $ABC$ and $ABCDE$, because their values could
depend on how we choose to compute them. This is one reason why, in the
definition of $AB$, we multiply entries of the $i$-th row of $A$ with entries
of the $j$-th column of $B$. Using rows both times would break associativity!

We shall give proofs of parts \textbf{(d)} and \textbf{(g)} of Proposition
\ref{prop.matrix-laws.1} in Section \ref{sect.intro.sum} below.

\subsection{Non-properties of matrix operations}

Conspicuously absent from Proposition \ref{prop.matrix-laws.1} is one
important law that is well-known to hold for numbers: commutativity of
multiplication (that is, $ab=ba$). This has a reason: it is false for
matrices. There are at least three reasons why it is false:

\begin{enumerate}
\item If $A$ and $B$ are matrices, then it can happen that $AB$ is
well-defined (i.e., $A$ has as many columns as $B$ has rows) but $BA$ is not
(i.e., $B$ does not have as many columns as $A$ has rows). For example, if
$A=\left(
\begin{array}
[c]{c}%
a\\
b\\
c
\end{array}
\right)  $ and $B=\left(
\begin{array}
[c]{cc}%
x & y
\end{array}
\right)  $, then $AB$ is well-defined but $BA$ is not.

\item If $A$ and $B$ are matrices such that both $AB$ and $BA$ well-defined,
then $AB$ and $BA$ might still have different dimensions. Namely, if $A$ is an
$n\times m$-matrix and $B$ is an $m\times n$-matrix, then $AB$ is an $n\times
n$-matrix, but $BA$ is an $m\times m$-matrix. So comparing $AB$ and $BA$ makes
no sense unless $n=m$.

\item Even if $AB$ and $BA$ are of the same dimensions, they can still be
distinct. For example, if $A=\left(
\begin{array}
[c]{cc}%
1 & 1\\
0 & 1
\end{array}
\right)  $ and $B=A^{T}=\left(
\begin{array}
[c]{cc}%
1 & 0\\
1 & 1
\end{array}
\right)  $, then $AB=\left(
\begin{array}
[c]{cc}%
2 & 1\\
1 & 1
\end{array}
\right)  $ whereas $BA=\left(
\begin{array}
[c]{cc}%
1 & 1\\
1 & 2
\end{array}
\right)  $.
\end{enumerate}

Two matrices $A$ and $B$ are said to \textit{commute} if $AB=BA$ (which, in
particular, means that both $AB$ and $BA$ are well-defined). You will
encounter many cases when matrices $A$ and $B$ happen to commute (for example,
every $n\times n$-matrix commutes with the $n\times n$ identity matrix; see
below for what this means); but in general there is no reason to expect two
randomly chosen matrices to commute.

As a consequence of matrices refusing to commute (in general), we cannot
reasonably define division of matrices. Actually, there are two reasons why we
cannot reasonably define division of matrices: First, if $A$ and $B$ are two
matrices, then it is not clear whether $\dfrac{A}{B}$ should mean a matrix $C$
satisfying $BC=A$, or a matrix $C$ satisfying $CB=A$. (The failure of
commutativity implies that these are two different things.) Second, in
general, neither of these matrices $C$ is necessarily unique; nor is it
guaranteed to exist. This is similar to the fact that we cannot divide by $0$
(in fact, $\dfrac{0}{0}$ would not be unique, while $\dfrac{1}{0}$ would not
exist); but with matrices, $0$ is not the only forbidden denominator. Here is
an example:

\begin{example}
\textbf{(a)} Let $A=\left(
\begin{array}
[c]{cc}%
1 & 0\\
0 & 0
\end{array}
\right)  $ and $B=A$. Then, $BC=A$ holds for $C=A$, but also for $C=\left(
\begin{array}
[c]{cc}%
1 & 0\\
0 & 1
\end{array}
\right)  $ (and also for many other matrices $C$). So the matrix $C$
satisfying $BC=A$ is not unique.

\textbf{(b)} Let $A=\left(
\begin{array}
[c]{cc}%
1 & 0\\
0 & 1
\end{array}
\right)  $ and $B=\left(
\begin{array}
[c]{cc}%
1 & 0\\
0 & 0
\end{array}
\right)  $. Then, there exists no matrix $C$ satisfing $BC=A$. Indeed, if
$C=\left(
\begin{array}
[c]{cc}%
x & y\\
z & w
\end{array}
\right)  $ is any matrix, then $BC=\left(
\begin{array}
[c]{cc}%
x & y\\
0 & 0
\end{array}
\right)  $ has its second row filled with zeroes, but $A$ does not; so $BC$
cannot equal $A$.
\end{example}

\subsection{\label{sect.intro.sum}(*) The summation sign}

We now take a break from studying matrices to introduce an important symbol:
the summation sign ($\sum$). This sign is one of the hallmarks of abstract
mathematics (and also computer science), and helps manipulate matrices comfortably.

\begin{definition}
\label{def.sum}Let $p$ and $q$ be two integers such that $p\leq q$. Let
$a_{p},a_{p+1},\ldots,a_{q}$ be some numbers. Then, $\sum_{k=p}^{q}a_{k}$
means the sum $a_{p}+a_{p+1}+\cdots+a_{q}$. The symbol $\sum$ is called the
summation sign; we pronounce the expression $\sum_{k=p}^{q}a_{k}$ as
\textquotedblleft sum of $a_{k}$ for all $k$ ranging from $p$ to
$q$\textquotedblright.
\end{definition}

Here are some examples:

\begin{itemize}
\item We have $\sum_{k=p}^{q}k=p+\left(  p+1\right)  +\cdots+q$. For example,
$\sum_{k=1}^{n}k=1+2+\cdots+n$. (A well-known formula says that this sum
$\sum_{k=1}^{n}k=1+2+\cdots+n$ equals $\dfrac{n\left(  n+1\right)  }{2}$. For
a concrete example, $\sum_{k=1}^{3}k=1+2+3=\dfrac{3\left(  3+1\right)  }{2}%
=6$.) For another example, $\sum_{k=-n}^{n}k=\left(  -n\right)  +\left(
-n+1\right)  +\cdots+n$. (This latter sum equals $0$, because it contains, for
each its addend, also its negative\footnote{except for the addend $0$, but
this $0$ doesn't change the sum anyway}.)

\item We have $\sum_{k=p}^{q}k^{2}=p^{2}+\left(  p+1\right)  ^{2}+\cdots
+q^{2}$. For example, $\sum_{k=1}^{n}k^{2}=1^{2}+2^{2}+\cdots+n^{2}$. (A
well-known formula says that this sum $\sum_{k=1}^{n}k^{2}=1^{2}+2^{2}%
+\cdots+n^{2}$ equals $\dfrac{n\left(  n+1\right)  \left(  2n+1\right)  }{6}$.)

\item If $p=q$, then $\sum_{k=p}^{q}a_{k}=a_{p}$. (A sum of only one number is
simply this number.)
\end{itemize}

Notice that the expression \textquotedblleft$\sum_{k=p}^{q}a_{k}%
$\textquotedblright\ is both a more compact and a more rigorous way to say
\textquotedblleft$a_{p}+a_{p+1}+\cdots+a_{q}$\textquotedblright. A computer
would not understand the expression \textquotedblleft$a_{p}+a_{p+1}%
+\cdots+a_{q}$\textquotedblright\ (it could only guess what the
\textquotedblleft$\cdots$\textquotedblright\ means, and computers are bad at
guessing); but the expression \textquotedblleft$\sum_{k=p}^{q}a_{k}%
$\textquotedblright\ has a well-defined meaning that can be rigorously defined
and can be explained to a computer\footnote{Of course, the way we defined it
above is not rigorous. Here is how to define it rigorously (if you know
recursion):
\par
\begin{itemize}
\item If $q=p$, then $\sum_{k=p}^{q}a_{k}$ is defined to be $a_{p}$.
\par
\item If $q>p$, then $\sum_{k=p}^{q}a_{k}$ is defined to be $\left(
\sum_{k=p}^{q-1}a_{k}\right)  +a_{q}$.
\end{itemize}
\par
This is a recursive definition (since it defines $\sum_{k=p}^{q}a_{k}$ in
terms of $\sum_{k=p}^{q-1}a_{k}$), and provides an algorithm to compute
$\sum_{k=p}^{q}a_{k}$.}. Thus, if you want to tell a computer to compute a
sum, the command you have to use will be closer to \textquotedblleft%
$\sum_{k=p}^{q}a_{k}$\textquotedblright\ than to \textquotedblleft%
$a_{p}+a_{p+1}+\cdots+a_{q}$\textquotedblright. For example, in Python, you
would have to write \textquotedblleft\texttt{sum(a[k] for k in range(p,
q+1))}\textquotedblright\ (where \textquotedblleft\texttt{a[k]}%
\textquotedblright\ is understood to return $a_{k}$)\ \ \ \ \footnote{Why
\textquotedblleft\texttt{q+1}\textquotedblright\ and not \textquotedblleft%
\texttt{q}\textquotedblright? Because Python defines \texttt{range(u, v)} as
the list $\left(  u,u+1,\ldots,v-1\right)  $ (that is, the list that starts at
$u$ and ends \textbf{just before} $v$). So \texttt{range(p, q)} would be
$\left(  p,p+1,\ldots,q-1\right)  $, but we want $\left(  p,p+1,\ldots
,q\right)  $.}.

A few remarks on the summation sign are in order:

\begin{definition}
\label{def.sum.2}\textbf{(a)} In the expression $\sum_{k=p}^{q}a_{k}$ (as
defined in Definition \ref{def.sum}), the letter $k$ is called the
\textit{summation index}. It doesn't have to be called $k$; any letter is
legitimate (as long as it is not already used otherwise). For example,
$\sum_{i=p}^{q}a_{i}$ and $\sum_{x=p}^{q}a_{x}$ are two synonymous ways to
write $\sum_{k=p}^{q}a_{k}$. Just make sure that the you are using the same
letter under the $\sum$ sign and to its right (so you should not write
$\sum_{i=p}^{q}a_{k}$, unless you mean the sum $\underbrace{a_{k}+a_{k}%
+\cdots+a_{k}}_{q-p+1\text{ times}}$).

\textbf{(b)} The sum $\sum_{k=p}^{q}a_{k}$ is also defined when $p>q$. In this
case, it is called an \textit{empty sum} (because there are no numbers
$a_{p},a_{p+1},\ldots,a_{q}$) and defined to be $0$. This convention might
sound somewhat arbitrary, but it is the most reasonable way to define empty sums.
\end{definition}

Using the summation sign, we can rewrite the product $AB$ of two matrices $A$
and $B$ (see Definition \ref{def.AB}) more nicely:

\begin{proposition}
\label{prop.AB}Let $n,m,p$ be three integers. Let $A$ be an $n\times
m$-matrix. Let $B$ be an $m\times p$-matrix. Then,%
\[
\left(  AB\right)  _{i,j}=\sum_{k=1}^{m}A_{i,k}B_{k,j}%
\ \ \ \ \ \ \ \ \ \ \text{for all }i\in\left\{  1,2,\ldots,n\right\}  \text{
and }j\in\left\{  1,2,\ldots,p\right\}  .
\]

\end{proposition}

\begin{proof}
[Proof of Proposition \ref{prop.AB}.]The definition of $AB$ yields%
\[
AB=\left(  A_{i,1}B_{1,j}+A_{i,2}B_{2,j}+\cdots+A_{i,m}B_{m,j}\right)  _{1\leq
i\leq n,\ 1\leq j\leq p}.
\]
Hence, for all $i\in\left\{  1,2,\ldots,n\right\}  $ and $j\in\left\{
1,2,\ldots,p\right\}  $, we have%
\[
\left(  AB\right)  _{i,j}=A_{i,1}B_{1,j}+A_{i,2}B_{2,j}+\cdots+A_{i,m}%
B_{m,j}=\sum_{k=1}^{m}A_{i,k}B_{k,j}%
\]
(because $\sum_{k=1}^{m}A_{i,k}B_{k,j}$ is exactly $A_{i,1}B_{1,j}%
+A_{i,2}B_{2,j}+\cdots+A_{i,m}B_{m,j}$, by its definition). Proposition
\ref{prop.AB} is proven.
\end{proof}

Here are two properties of sums that are fairly clear if you understand how
sums are defined:

\begin{proposition}
\label{prop.sum.dist}Let $p$ and $q$ be two integers such that $p\leq q$. Let
$a_{p},a_{p+1},\ldots,a_{q}$ be some numbers. Let $b$ be a number. Then,
\[
\sum_{k=p}^{q}ba_{k}=b\sum_{k=p}^{q}a_{k}.
\]
(The expression $\sum_{k=p}^{q}ba_{k}$ has to be read as $\sum_{k=p}%
^{q}\left(  ba_{k}\right)  $.)
\end{proposition}

\begin{proof}
[Proof of Proposition \ref{prop.sum.dist}.]By the definition of $\sum$, we
have%
\begin{align*}
\sum_{k=p}^{q}ba_{k}  &  =ba_{p}+ba_{p+1}+\cdots+ba_{q}\\
&  =b\underbrace{\left(  a_{p}+a_{p+1}+\cdots+a_{q}\right)  }_{\substack{=\sum
_{k=p}^{q}a_{k}\\\text{(by the definition of }\sum_{k=p}^{q}a_{k}\text{)}%
}}=b\sum_{k=p}^{q}a_{k}.
\end{align*}

\end{proof}

\begin{proposition}
\label{prop.sum.a+b}Let $p$ and $q$ be two integers such that $p\leq q$. Let
$a_{p},a_{p+1},\ldots,a_{q}$ be some numbers. Let $b_{p},b_{p+1},\ldots,b_{q}$
be some numbers. Then,
\[
\sum_{k=p}^{q}\left(  a_{k}+b_{k}\right)  =\sum_{k=p}^{q}a_{k}+\sum_{k=p}%
^{q}b_{k}.
\]
(The expression $\sum_{k=p}^{q}a_{k}+\sum_{k=p}^{q}b_{k}$ has to be read as
$\left(  \sum_{k=p}^{q}a_{k}\right)  +\left(  \sum_{k=p}^{q}b_{k}\right)  $.)
\end{proposition}

\begin{proof}
[Proof of Proposition \ref{prop.sum.dist}.]By the definition of $\sum$, we
have%
\begin{align*}
\sum_{k=p}^{q}\left(  a_{k}+b_{k}\right)   &  =\left(  a_{p}+b_{p}\right)
+\left(  a_{p+1}+b_{p+1}\right)  +\cdots+\left(  a_{q}+b_{q}\right) \\
&  =\underbrace{\left(  a_{p}+a_{p+1}+\cdots+a_{q}\right)  }_{\substack{=\sum
_{k=p}^{q}a_{k}\\\text{(by the definition of }\sum_{k=p}^{q}a_{k}\text{)}%
}}+\underbrace{\left(  b_{p}+b_{p+1}+\cdots+b_{q}\right)  }_{\substack{=\sum
_{k=p}^{q}b_{k}\\\text{(by the definition of }\sum_{k=p}^{q}b_{k}\text{)}}}\\
&  =\sum_{k=p}^{q}a_{k}+\sum_{k=p}^{q}b_{k}.
\end{align*}

\end{proof}

Our goal in this section is to prove Proposition \ref{prop.matrix-laws.1}
\textbf{(g)}, illustrating the use and manipulation of the $\sum$ sign.
However, as a warmup, let us first prove Proposition \ref{prop.matrix-laws.1}
\textbf{(d)} (which is simple enough that you can easily check it without
$\sum$ signs, but is nevertheless worth proving using the $\sum$ sign just to
demonstrate how to work with the $\sum$ sign):

\begin{proof}
[Proof of Proposition \ref{prop.matrix-laws.1} \textbf{(d)}.]Let $A$ be an
$n\times m$-matrix. Let $B$ and $C$ be two $m\times p$-matrices.

We shall show that $\left(  A\left(  B+C\right)  \right)  _{i,j}=\left(
AB+AC\right)  _{i,j}$ for all $i\in\left\{  1,2,\ldots,n\right\}  $ and
$j\in\left\{  1,2,\ldots,p\right\}  $. Once this is proven, this will entail
that corresponding entries of the two $n\times p$-matrices $A\left(
B+C\right)  $ and $AB+AC$ are equal; and thus, these two matrices have to be equal.

Proposition \ref{prop.AB} yields%
\begin{equation}
\left(  AB\right)  _{i,j}=\sum_{k=1}^{m}A_{i,k}B_{k,j}
\label{pf.prop.matrix-laws.1.d.1}%
\end{equation}
for all $i\in\left\{  1,2,\ldots,n\right\}  $ and $j\in\left\{  1,2,\ldots
,p\right\}  $.

Proposition \ref{prop.AB} (applied to $C$ instead of $B$) yields%
\begin{equation}
\left(  AC\right)  _{i,j}=\sum_{k=1}^{m}A_{i,k}C_{k,j}
\label{pf.prop.matrix-laws.1.d.2}%
\end{equation}
for all $i\in\left\{  1,2,\ldots,n\right\}  $ and $j\in\left\{  1,2,\ldots
,p\right\}  $.

Finally, Proposition \ref{prop.AB} (applied to $B+C$ instead of $B$) yields%
\begin{align*}
\left(  A\left(  B+C\right)  \right)  _{i,j}  &  =\sum_{k=1}^{m}%
A_{i,k}\underbrace{\left(  B+C\right)  _{k,j}}_{\substack{=B_{k,j}%
+C_{k,j}\\\text{(since matrices are added entry by entry)}}}=\sum_{k=1}%
^{m}\underbrace{A_{i,k}\left(  B_{k,j}+C_{k,j}\right)  }_{=A_{i,k}%
B_{k,j}+A_{i,k}C_{k,j}}\\
&  =\sum_{k=1}^{m}\left(  A_{i,k}B_{k,j}+A_{i,k}C_{k,j}\right)
=\underbrace{\sum_{k=1}^{m}A_{i,k}B_{k,j}}_{\substack{=\left(  AB\right)
_{i,j}\\\text{(by (\ref{pf.prop.matrix-laws.1.d.1}))}}}+\underbrace{\sum
_{k=1}^{m}A_{i,k}C_{k,j}}_{\substack{=\left(  AC\right)  _{i,j}\\\text{(by
(\ref{pf.prop.matrix-laws.1.d.2}))}}}\\
&  \ \ \ \ \ \ \ \ \ \ \left(
\begin{array}
[c]{c}%
\text{by Proposition \ref{prop.sum.a+b}, applied to }p=1\text{, }q=m\text{,}\\
a_{k}=A_{i,k}B_{k,j}\text{ and }b_{k}=A_{i,k}C_{k,j}%
\end{array}
\right) \\
&  =\left(  AB\right)  _{i,j}+\left(  AC\right)  _{i,j}=\left(  AB+AC\right)
_{i,j}\ \ \ \ \ \ \ \ \ \ \left(
\begin{array}
[c]{c}%
\text{again because matrices}\\
\text{are added entry by entry}%
\end{array}
\right)
\end{align*}
for all $i\in\left\{  1,2,\ldots,n\right\}  $ and $j\in\left\{  1,2,\ldots
,p\right\}  $. In other words, each entry of the $n\times p$-matrix $A\left(
B+C\right)  $ equals the corresponding entry of $AB+AC$. Thus, the matrix
$A\left(  B+C\right)  $ equals $AB+AC$. This proves Proposition
\ref{prop.matrix-laws.1} \textbf{(d)}.
\end{proof}

Before we can prove Proposition \ref{prop.matrix-laws.1} \textbf{(g)}, we need
another fact about sums:

\begin{proposition}
\label{prop.sum.ij}Let $m\in\mathbb{N}$ and $p\in\mathbb{N}$. Assume that a
number $a_{k,\ell}$ is given for every $k\in\left\{  1,2,\ldots,m\right\}  $
and $\ell\in\left\{  1,2,\ldots,p\right\}  $. Then,%
\[
\sum_{k=1}^{m}\sum_{\ell=1}^{p}a_{k,\ell}=\sum_{\ell=1}^{p}\sum_{k=1}%
^{m}a_{k,\ell}.
\]
(Note that an expression like $\sum_{k=1}^{m}\sum_{\ell=1}^{p}a_{k,\ell}$ has
to be understood as $\sum_{k=1}^{m}\left(  \sum_{\ell=1}^{p}a_{k,\ell}\right)
$. It is a \textquotedblleft nested sum\textquotedblright, i.e., a sum of
sums. For example,%
\begin{align*}
\sum_{k=1}^{3}\underbrace{\sum_{\ell=1}^{4}k\cdot\ell}_{=k\cdot1+k\cdot
2+k\cdot3+k\cdot4} &  =\sum_{k=1}^{3}\left(  k\cdot1+k\cdot2+k\cdot
3+k\cdot4\right)  \\
&  =\left(  1\cdot1+1\cdot2+1\cdot3+1\cdot4\right)  +\left(  2\cdot
1+2\cdot2+2\cdot3+2\cdot4\right)  s\\
&  \ \ \ \ \ \ \ \ \ \ +\left(  3\cdot1+3\cdot2+3\cdot3+3\cdot4\right)  .
\end{align*}
)
\end{proposition}

\begin{example}
For $m=2$ and $p=3$, Proposition \ref{prop.sum.ij} says that%
\[
\sum_{k=1}^{2}\sum_{\ell=1}^{3}a_{k,\ell}=\sum_{\ell=1}^{3}\sum_{k=1}%
^{2}a_{k,\ell}.
\]
In other words,%
\[
\left(  a_{1,1}+a_{1,2}+a_{1,3}\right)  +\left(  a_{2,1}+a_{2,2}%
+a_{2,3}\right)  =\left(  a_{1,1}+a_{2,1}\right)  +\left(  a_{1,2}%
+a_{2,2}\right)  +\left(  a_{1,3}+a_{2,3}\right)  .
\]

\end{example}

\begin{proof}
[Proof of Proposition \ref{prop.sum.ij}.]Comparing%
\begin{align*}
&  \sum_{k=1}^{m}\underbrace{\sum_{\ell=1}^{p}a_{k,\ell}}_{\substack{=a_{k,1}%
+a_{k,2}+\cdots+a_{k,p}\\\text{(by the definition of the }\sum\text{ sign)}%
}}\\
&  =\sum_{k=1}^{m}\left(  a_{k,1}+a_{k,2}+\cdots+a_{k,p}\right)  \\
&  =\left(  a_{1,1}+a_{1,2}+\cdots+a_{1,p}\right)  +\left(  a_{2,1}%
+a_{2,2}+\cdots+a_{2,p}\right)  +\cdots+\left(  a_{m,1}+a_{m,2}+\cdots
+a_{m,p}\right)  \\
&  \ \ \ \ \ \ \ \ \ \ \left(  \text{by the definition of the }\sum\text{
sign}\right)  \\
&  =\left(  \text{the sum of all possible numbers }a_{k,\ell}\right)
\end{align*}
with%
\begin{align*}
&  \sum_{\ell=1}^{p}\underbrace{\sum_{k=1}^{m}a_{k,\ell}}%
_{\substack{=a_{1,\ell}+a_{2,\ell}+\cdots+a_{m,\ell}\\\text{(by the definition
of the }\sum\text{ sign)}}}\\
&  =\sum_{\ell=1}^{p}\left(  a_{1,\ell}+a_{2,\ell}+\cdots+a_{m,\ell}\right)
\\
&  =\left(  a_{1,1}+a_{2,1}+\cdots+a_{m,1}\right)  +\left(  a_{1,2}%
+a_{2,2}+\cdots+a_{m,2}\right)  +\cdots+\left(  a_{1,p}+a_{2,p}+\cdots
+a_{m,p}\right)  \\
&  \ \ \ \ \ \ \ \ \ \ \left(  \text{by the definition of the }\sum\text{
sign}\right)  \\
&  =\left(  \text{the sum of all possible numbers }a_{k,\ell}\right)  ,
\end{align*}
we obtain $\sum_{k=1}^{m}\left(  \sum_{\ell=1}^{p}a_{k,\ell}\right)
=\sum_{\ell=1}^{p}\left(  \sum_{k=1}^{m}a_{k,\ell}\right)  $.

(A more rigorous proof could be given using induction; but this one will
suffice for our purposes. Notice the visual meaning of the above proof: If we
place the $mp$ numbers $a_{k,\ell}$ into a matrix $\left(  a_{k,\ell}\right)
_{1\leq k\leq m,\ 1\leq\ell\leq p}$, then

\begin{itemize}
\item the number $\sum_{k=1}^{m}\sum_{\ell=1}^{p}a_{k,\ell}$ is obtained by
summing the entries in each row of the matrix, and then summing the resulting sums;

\item the number $\sum_{\ell=1}^{p}\sum_{k=1}^{m}a_{k,\ell}$ is obtained by
summing the entries in each column of the matrix, and then summing the
resulting sums.
\end{itemize}

Thus, clearly, both numbers are equal (namely, equal to the sum of all entries
of the matrix).)
\end{proof}

Now, we can prove Proposition \ref{prop.matrix-laws.1} \textbf{(g)}:

\begin{proof}
[Proof of Proposition \ref{prop.matrix-laws.1} \textbf{(g)}.]Let $A$ be an
$n\times m$-matrix. Let $B$ be an $m\times p$-matrix. Let $C$ be a $p\times q$-matrix.

We must show that $A\left(  BC\right)  =\left(  AB\right)  C$. In order to do
so, it suffices to show that $\left(  A\left(  BC\right)  \right)
_{i,j}=\left(  \left(  AB\right)  C\right)  _{i,j}$ for all $i\in\left\{
1,2,\ldots,n\right\}  $ and $j\in\left\{  1,2,\ldots,q\right\}  $ (because
this will show that respective entries of the two $n\times q$-matrices
$A\left(  BC\right)  $ and $\left(  AB\right)  C$ are equal, and thus the two
matrices are equal).

We know that $A$ is an $n\times m$-matrix, and that $BC$ is an $m\times
q$-matrix. Hence, we can apply Proposition \ref{prop.AB} to $n$, $m$, $q$, $A$
and $BC$ instead of $n$, $m$, $p$, $A$ and $B$. We thus obtain%
\begin{equation}
\left(  A\left(  BC\right)  \right)  _{i,j}=\sum_{k=1}^{m}A_{i,k}\left(
BC\right)  _{k,j} \label{pf.prop.matrix-laws.1.g.A(BC)}%
\end{equation}
for all $i\in\left\{  1,2,\ldots,n\right\}  $ and $j\in\left\{  1,2,\ldots
,q\right\}  $. Also, we can apply Proposition \ref{prop.AB} to $m$, $p$, $q$,
$B$ and $C$ instead of $n$, $m$, $p$, $A$ and $B$. We thus obtain%
\begin{align}
\left(  BC\right)  _{i,j}  &  =\sum_{k=1}^{p}B_{i,k}C_{k,j}=\sum_{\ell=1}%
^{p}B_{i,\ell}C_{\ell,j}\label{pf.prop.matrix-laws.1.g.BC}\\
&  \ \ \ \ \ \ \ \ \ \ \left(  \text{here, we renamed the summation index
}k\text{ as }\ell\right) \nonumber
\end{align}
for all $i\in\left\{  1,2,\ldots,m\right\}  $ and $j\in\left\{  1,2,\ldots
,p\right\}  $. Furthermore, we can apply Proposition \ref{prop.AB} to $n$,
$p$, $q$, $AB$ and $C$ instead of $n$, $m$, $p$, $A$ and $B$. We thus find%
\begin{align}
\left(  \left(  AB\right)  C\right)  _{i,j}  &  =\sum_{k=1}^{p}\left(
AB\right)  _{i,k}C_{k,j}=\sum_{\ell=1}^{p}\left(  AB\right)  _{i,\ell}%
C_{\ell,j}\label{pf.prop.matrix-laws.1.g.(AB)C}\\
&  \ \ \ \ \ \ \ \ \ \ \left(  \text{here, we renamed the summation index
}k\text{ as }\ell\right) \nonumber
\end{align}
for all $i\in\left\{  1,2,\ldots,n\right\}  $ and $j\in\left\{  1,2,\ldots
,q\right\}  $. Finally, Proposition \ref{prop.AB} (applied verbatim) yields%
\begin{equation}
\left(  AB\right)  _{i,j}=\sum_{k=1}^{m}A_{i,k}B_{k,j}
\label{pf.prop.matrix-laws.1.g.AB}%
\end{equation}
for all $i\in\left\{  1,2,\ldots,n\right\}  $ and $j\in\left\{  1,2,\ldots
,p\right\}  $.

Now that we have found formulas for the entries of all matrices involved, we
can perform our computation: For all $i\in\left\{  1,2,\ldots,n\right\}  $ and
$j\in\left\{  1,2,\ldots,q\right\}  $, we have%
\begin{align*}
\left(  A\left(  BC\right)  \right)  _{i,j}  &  =\sum_{k=1}^{m}A_{i,k}%
\underbrace{\left(  BC\right)  _{k,j}}_{\substack{=\sum_{\ell=1}^{p}B_{k,\ell
}C_{\ell,j}\\\text{(by (\ref{pf.prop.matrix-laws.1.g.BC}), applied to }k\text{
instead of }i\text{)}}}\ \ \ \ \ \ \ \ \ \ \left(  \text{by
(\ref{pf.prop.matrix-laws.1.g.(AB)C})}\right) \\
&  =\sum_{k=1}^{m}\underbrace{A_{i,k}\left(  \sum_{\ell=1}^{p}B_{k,\ell
}C_{\ell,j}\right)  }_{\substack{=\sum_{\ell=1}^{p}A_{i,k}B_{k,\ell}C_{\ell
,j}\\\text{(by an application of}\\\text{Proposition \ref{prop.sum.dist})}%
}}=\sum_{k=1}^{m}\sum_{\ell=1}^{p}A_{i,k}B_{k,\ell}C_{\ell,j}\\
&  =\sum_{\ell=1}^{p}\sum_{k=1}^{m}A_{i,k}B_{k,\ell}C_{\ell,j}%
\ \ \ \ \ \ \ \ \ \ \left(  \text{by Proposition \ref{prop.sum.ij}, applied to
}a_{k,\ell}=A_{i,k}B_{k,\ell}C_{\ell,j}\right)
\end{align*}
and%
\begin{align*}
\left(  \left(  AB\right)  C\right)  _{i,j}  &  =\sum_{\ell=1}^{p}%
\underbrace{\left(  AB\right)  _{i,\ell}}_{\substack{=\sum_{k=1}^{m}%
A_{i,k}B_{k,\ell}\\\text{(by (\ref{pf.prop.matrix-laws.1.g.AB}), applied to
}\ell\\\text{instead of }j\text{)}}}C_{\ell,j}\ \ \ \ \ \ \ \ \ \ \left(
\text{by (\ref{pf.prop.matrix-laws.1.g.A(BC)})}\right) \\
&  =\sum_{k=1}^{m}\underbrace{\left(  \sum_{k=1}^{m}A_{i,k}B_{k,\ell}\right)
C_{\ell,j}}_{\substack{=C_{\ell,j}\sum_{k=1}^{m}A_{i,k}B_{k,\ell}=\sum
_{\ell=1}^{p}C_{\ell,j}A_{i,k}B_{k,\ell}\\\text{(by an application
of}\\\text{Proposition \ref{prop.sum.dist})}}}=\sum_{k=1}^{m}\sum_{\ell=1}%
^{p}\underbrace{C_{\ell,j}A_{i,k}B_{k,\ell}}_{=A_{i,k}B_{k,\ell}C_{\ell,j}}\\
&  =\sum_{\ell=1}^{p}\sum_{k=1}^{m}A_{i,k}B_{k,\ell}C_{\ell,j}.
\end{align*}
Comparing these two equalities, we obtain
\[
\left(  A\left(  BC\right)  \right)  _{i,j}=\left(  \left(  AB\right)
C\right)  _{i,j}%
\]
for all $i\in\left\{  1,2,\ldots,n\right\}  $ and $j\in\left\{  1,2,\ldots
,q\right\}  $. In other words, each entry of the matrix $A\left(  BC\right)  $
equals the corresponding entry of the matrix $\left(  AB\right)  C$. Thus, the
matrices $A\left(  BC\right)  $ and $\left(  AB\right)  C$ are equal. This
proves Proposition \ref{prop.matrix-laws.1} \textbf{(g)}.
\end{proof}

We have just proven the hardest part of Proposition \ref{prop.matrix-laws.1}.
The rest is fairly straightforward.

\subsection{The zero matrix}

\begin{definition}
Let $n\in\mathbb{N}$ and $m\in\mathbb{N}$. Then, the $n\times m$\textit{ zero
matrix} means the matrix $\left(  0\right)  _{1\leq i\leq n,\ 1\leq j\leq m}$.
This is the $n\times m$ matrix filled with zeroes. It is called $0_{n\times
m}$. (When no confusion with the number $0$ can arise, we will just call it
$0$.)
\end{definition}

For example, the $2\times3$ zero matrix is $\left(
\begin{array}
[c]{ccc}%
0 & 0 & 0\\
0 & 0 & 0
\end{array}
\right)  $.

The zero matrix behaves very much like the number $0$:

\begin{proposition}
\label{prop.matrix-laws.0}Let $n\in\mathbb{N}$ and $m\in\mathbb{N}$. Then:

\textbf{(a)} We have $0_{n\times m}+A=A+0_{n\times m}=A$ for each $n\times
m$-matrix $A$.

\textbf{(b)} We have $0_{n\times m}A=0_{n\times p}$ for each $p\in\mathbb{N}$
and each $m\times p$-matrix $A$.

\textbf{(c)} We have $A0_{n\times m}=0_{p\times m}$ for each $p\in\mathbb{N}$
and each $p\times n$-matrix $A$.

\textbf{(d)} We have $0A=0_{n\times m}$ for each $n\times m$-matrix $A$.
\end{proposition}

Numbers are known to be zero-divisor-free: If a product $ab$ of two numbers
$a$ and $b$ is $0$, then one of $a$ and $b$ must be $0$. This also fails for
matrices: If $A=\left(
\begin{array}
[c]{cc}%
1 & 0\\
0 & 0
\end{array}
\right)  $ and $B=\left(
\begin{array}
[c]{cc}%
0 & 0\\
0 & 1
\end{array}
\right)  $, then $AB=0_{2\times2}$ is the zero matrix, although neither $A$
nor $B$ is the zero matrix.

\subsection{The identity matrix}

\begin{definition}
Let $n\in\mathbb{N}$. The \textit{diagonal entries} of an $n\times n$-matrix
$A$ are its entries $A_{1,1},A_{2,2},\ldots,A_{n,n}$. In other words, they are
the entries $A_{i,j}$ for $i=j$.
\end{definition}

For example, the diagonal entries of $\left(
\begin{array}
[c]{cc}%
a & b\\
c & d
\end{array}
\right)  $ are $a$ and $d$. The name \textquotedblleft diagonal
entries\textquotedblright\ comes from the visualization of an $n\times
n$-matrix as a square table: When we say \textquotedblleft
diagonal\textquotedblright, we always mean the diagonal of the square that
connects the upper-left corner with the lower-right corner; the diagonal
entries are simply the entries along this diagonal. (The other diagonal is
called the \textquotedblleft antidiagonal\textquotedblright\ in linear algebra.)

\begin{definition}
Let $n\in\mathbb{N}$. Then, the $n\times n$ \textit{identity matrix} means the
matrix $\left(  \delta_{i,j}\right)  _{1\leq i\leq n,\ 1\leq j\leq n}$, where%
\begin{equation}
\delta_{i,j}=%
\begin{cases}
1, & \text{if }i=j;\\
0, & \text{if }i\neq j
\end{cases}
\ \ \ \ \ \ \ \ \ \ \text{for all }i\text{ and }j. \label{eq.def.In.eq}%
\end{equation}
This is the $n\times n$-matrix whose diagonal entries all equal $1$, and whose
all other entries equal $0$. It is denoted by $I_{n}$. (Other people call it
$I$ or $E$ or $E_{n}$.)
\end{definition}

For example, the $3\times3$ identity matrix is $I_{3}=\left(
\begin{array}
[c]{ccc}%
1 & 0 & 0\\
0 & 1 & 0\\
0 & 0 & 1
\end{array}
\right)  $.

The notation $\delta_{i,j}$ defined in (\ref{eq.def.In.eq}) is called the
\textit{Kronecker delta}; it is extremely simple and yet highly useful.

The identity matrix behaves very much like the number $1$:

\begin{proposition}
\label{prop.matrix-laws.id}Let $n\in\mathbb{N}$ and $m\in\mathbb{N}$.

\textbf{(a)} We have $I_{n}A=A$ for each $n\times m$-matrix $A$.

\textbf{(b)} We have $AI_{m}=A$ for each $n\times m$-matrix $A$.
\end{proposition}

\subsection{(*) Proof of $AI_{n}=A$}

Let me give a proof of Proposition \ref{prop.matrix-laws.id} \textbf{(b)}, to
illustrate the following simple, yet important point about summations and the
Kronecker delta\footnote{This is something that often comes up in computations
(particularly in physics and computer science).}:

\begin{proposition}
\label{prop.sum.delta}Let $p$ and $q$ be two integers such that $p\leq q$. Let
$r\in\left\{  p,p+1,\ldots,q\right\}  $. Let $a_{p},a_{p+1},\ldots,a_{q}$ be
some numbers. Then,%
\[
\sum_{k=p}^{q}a_{k}\delta_{k,r}=a_{r}.
\]

\end{proposition}

\begin{example}
For $p=1$, $q=5$ and $r=4$, Proposition \ref{prop.sum.delta} says that
$\sum_{k=1}^{5}a_{k}\delta_{k,4}=a_{4}$. This is easy to check:%
\begin{align*}
\sum_{k=1}^{5}a_{k}\delta_{k,4}  &  =a_{1}\underbrace{\delta_{1,4}%
}_{\substack{=0\\\text{(since }1\neq4\text{)}}}+a_{2}\underbrace{\delta_{2,4}%
}_{\substack{=0\\\text{(since }2\neq4\text{)}}}+a_{3}\underbrace{\delta_{3,4}%
}_{\substack{=0\\\text{(since }3\neq4\text{)}}}+a_{4}\underbrace{\delta_{4,4}%
}_{\substack{=1\\\text{(since }4=4\text{)}}}+a_{5}\underbrace{\delta_{5,4}%
}_{\substack{=0\\\text{(since }5\neq4\text{)}}}\\
&  =a_{1}0+a_{2}0+a_{3}0+a_{4}1+a_{5}0=a_{4}1=a_{4}.
\end{align*}
What you should see on this example is that all but one addends of the sum
$\sum_{k=p}^{q}a_{k}\delta_{k,r}$ are zero, and the remaining one addend is
$a_{r}\underbrace{\delta_{r,r}}_{=1}=a_{r}1=a_{r}$. The proof below is just
writing this down in the general situation.
\end{example}

\begin{proof}
[Proof of Proposition \ref{prop.sum.delta}.]By the definition of the $\sum$
sign, we have%
\begin{align*}
\sum_{k=p}^{q}a_{k}\delta_{k,r}  &  =a_{p}\delta_{p,r}+a_{p+1}\delta
_{p+1,r}+\cdots+a_{q}\delta_{q,r}\\
&  =\left(  \text{the sum of all terms of the form }a_{k}\delta_{k,r}\right)
\\
&  =a_{r}\underbrace{\delta_{r,r}}_{\substack{=1\\\text{(since }r=r\text{)}%
}}+\underbrace{\left(  \text{the sum of all terms of the form }a_{k}%
\delta_{k,r}\text{ with }k\neq r\right)  }_{\substack{=0\\\text{(since all
terms of the form }a_{k}\delta_{k,r}\text{ with }k\neq r\text{ are
}0\\\text{(because if }k\neq r\text{, then }a_{k}\underbrace{\delta_{k,r}%
}_{\substack{=0\\\text{(since }k\neq r\text{)}}}=a_{k}0=0\text{))}}}\\
&  \ \ \ \ \ \ \ \ \ \ \left(  \text{here, we have pulled out the addend
}a_{r}\delta_{r,r}\text{ out of the sum}\right) \\
&  =a_{r}1+0=a_{r}.
\end{align*}

\end{proof}

\begin{proof}
[Proof of Proposition \ref{prop.matrix-laws.id} \textbf{(b)}.]We have
$I_{m}=\left(  \delta_{i,j}\right)  _{1\leq i\leq m,\ 1\leq j\leq m}$ (this is
how we defined $I_{m}$), and thus%
\begin{equation}
\left(  I_{m}\right)  _{u,v}=\delta_{u,v}\ \ \ \ \ \ \ \ \ \ \text{for all
}u\in\left\{  1,2,\ldots,m\right\}  \text{ and }v\in\left\{  1,2,\ldots
,m\right\}  . \label{pf.prop.matrix-laws.id.b.Im}%
\end{equation}


Let $A$ be an $n\times m$-matrix. For every $i\in\left\{  1,2,\ldots
,n\right\}  $ and $j\in\left\{  1,2,\ldots,m\right\}  $, we have%
\begin{align*}
\left(  AI_{m}\right)  _{i,j}  &  =\sum_{k=1}^{m}A_{i,k}\underbrace{\left(
I_{m}\right)  _{k,j}}_{\substack{=\delta_{k,j}\\\text{(by
(\ref{pf.prop.matrix-laws.id.b.Im}))}}}\ \ \ \ \ \ \ \ \ \ \left(  \text{by
Proposition \ref{prop.AB}, applied to }p=m\text{ and }B=I_{m}\right) \\
&  =\sum_{k=1}^{m}A_{i,k}\delta_{k,j}=A_{i,j}%
\end{align*}
(by Proposition \ref{prop.sum.delta}, applied to $p=1$, $q=m$, $r=j$ and
$a_{k}=A_{i,k}$). In other words, each entry of the $n\times m$-matrix
$AI_{m}$ equals the corresponding entry of the $n\times m$-matrix $A$. In
other words, $AI_{m}$ equals $A$. This proves Proposition
\ref{prop.matrix-laws.id} \textbf{(b)}.
\end{proof}

Proposition \ref{prop.matrix-laws.id} \textbf{(a)} can be proven similarly
(but this time, instead of the sum $\sum_{k=1}^{m}A_{i,k}\delta_{k,j}$, we
must consider the sum $\sum_{k=1}^{m}\delta_{i,k}A_{k,j}=\sum_{k=1}^{m}%
A_{k,j}\delta_{i,k}=A_{i,j}$).

\subsection{Powers of a matrix}

The $k$-th power of a number $a$ (where $k\in\mathbb{N}$) is defined by
repeated multiplication: We start with $a^{0}=1$\ \ \ \ \footnote{Yes, this is
how $a^{0}$ is defined, for all $a$. Anyone who tells you that the number
$0^{0}$ is undefined is merely spreading their confusion.}, and we define each
next power of $a$ by multiplying the previous one by $a$. In formulas:
$a^{k+1}=a\cdot a^{k}$ for each $k\in\mathbb{N}$. Thus,%
\begin{align*}
a^{1}  &  =a\cdot\underbrace{a^{0}}_{=1}=a\cdot1=a;\\
a^{2}  &  =a\cdot\underbrace{a^{1}}_{=a}=a\cdot a;\\
a^{3}  &  =a\cdot\underbrace{a^{2}}_{=a\cdot a}=a\cdot a\cdot a,
\end{align*}
etc.. We can explicitly write
\[
a^{k}=\underbrace{a\cdot a\cdot\cdots\cdot a}_{k\text{ times }a}%
\ \ \ \ \ \ \ \ \ \ \text{for each }k\in\mathbb{N},
\]
where we understand $\underbrace{a\cdot a\cdot\cdots\cdot a}_{0\text{ times
}a}$ to mean $1$\ \ \ \ \footnote{This is a standard convention: An empty
product of numbers always means $1$.}.

We can play the same game with square matrices, but instead of the number $1$
we now take the $n\times n$ identity matrix $I_{n}$:

\begin{definition}
Let $n\in\mathbb{N}$. Let $A$ be an $n\times n$-matrix. Then, the $k$-th power
of the matrix $A$ (where $k\in\mathbb{N}$) is defined by repeated
multiplication: We start with $A^{0}=I_{n}$, and we define each next power of
$A$ by multiplying the previous one by $A$. In formulas: $A^{k+1}=A\cdot
A^{k}$ for each $k\in\mathbb{N}$. Explicitly, $A^{k}=\underbrace{A\cdot
A\cdot\cdots\cdot A}_{k\text{ times }A}$ for each $k\in\mathbb{N}$, where the
empty product means $I_{n}$.
\end{definition}

Notice that we have been a bit sloppy when we said \textquotedblleft
multiplying the previous one by $A$\textquotedblright: When we multiply a
matrix $B$ by $A$, we might mean either $AB$ or $BA$, and as we know, these
two products can be different (matrices don't always commute!). However, in
the above definition, this makes no matter, because both definitions lead to
the same explicit formula $A^{k}=\underbrace{A\cdot A\cdot\cdots\cdot
A}_{k\text{ times }A}$ (which is well-defined because of general associativity).

We now have a first moderately interesting example of commuting matrices: Any
two powers of a square matrix commute. (In other words: For any $n\times
n$-matrix $A$, and any $u\in\mathbb{N}$ and $v\in\mathbb{N}$, the two matrices
$A^{u}$ and $A^{v}$ commute. This follows by observing that $A^{u}%
A^{v}=A^{u+v}=A^{v}A^{u}$.)

\section{Gaussian elimination}

\subsection{Linear equations and matrices}

We now take aim at understanding Gaussian elimination in terms of matrices.
First of all, let us see what solving linear equations has to do with matrices.

\begin{example}
\label{exam.systems}Consider the following system of equations in three
unknowns $x,y,z$:%
\begin{equation}
\left\{
\begin{array}
[c]{c}%
3x+6y-z=2;\\
7x+4y-3z=3;\\
-y+8z=1
\end{array}
\right.  . \label{eq.exam.systems.1}%
\end{equation}
I claim that this system of equations is equivalent to the single equation%
\begin{equation}
\left(
\begin{array}
[c]{ccc}%
3 & 6 & -1\\
7 & 4 & -3\\
0 & -1 & 8
\end{array}
\right)  \left(
\begin{array}
[c]{c}%
x\\
y\\
z
\end{array}
\right)  =\left(
\begin{array}
[c]{c}%
2\\
3\\
1
\end{array}
\right)  \label{eq.exam.systems.2}%
\end{equation}
(this is an equation between two column vectors of size $3$, so no wonder that
it encodes a whole system of linear equations).

Why are (\ref{eq.exam.systems.1}) and (\ref{eq.exam.systems.2}) equivalent?
Well, the left hand side of (\ref{eq.exam.systems.2}) is%
\[
\left(
\begin{array}
[c]{ccc}%
3 & 6 & -1\\
7 & 4 & -3\\
0 & -1 & 8
\end{array}
\right)  \left(
\begin{array}
[c]{c}%
x\\
y\\
z
\end{array}
\right)  =\left(
\begin{array}
[c]{c}%
3x+6y+\left(  -1\right)  z\\
7x+4y+\left(  -3\right)  z\\
0x+\left(  -1\right)  y+8z
\end{array}
\right)  =\left(
\begin{array}
[c]{c}%
3x+6y-z\\
7x+4y-3z\\
-y+8z
\end{array}
\right)  .
\]
Thus, (\ref{eq.exam.systems.2}) is equivalent to%
\[
\left(
\begin{array}
[c]{c}%
3x+6y-z\\
7x+4y-3z\\
-y+8z
\end{array}
\right)  =\left(
\begin{array}
[c]{c}%
2\\
3\\
1
\end{array}
\right)  .
\]
But this is clearly equivalent to (\ref{eq.exam.systems.1}).
\end{example}

More generally, the system of $m$ linear equations
\[
\left\{
\begin{array}
[c]{c}%
a_{1,1}x_{1}+a_{1,2}x_{2}+\cdots+a_{1,n}x_{n}=b_{1};\\
a_{2,1}x_{1}+a_{2,2}x_{2}+\cdots+a_{2,n}x_{n}=b_{2};\\
\vdots\\
a_{m,1}x_{1}+a_{m,2}x_{2}+\cdots+a_{m,n}x_{n}=b_{m}%
\end{array}
\right.
\]
in $n$ unknowns $x_{1},x_{2},\ldots,x_{n}$ is equivalent to the vector
equation%
\begin{equation}
\left(
\begin{array}
[c]{cccc}%
a_{1,1} & a_{1,2} & \cdots & a_{1,n}\\
a_{2,1} & a_{2,2} & \cdots & a_{2,n}\\
\vdots & \vdots & \ddots & \vdots\\
a_{m,1} & a_{m,2} & \cdots & a_{m,n}%
\end{array}
\right)  \left(
\begin{array}
[c]{c}%
x_{1}\\
x_{2}\\
\vdots\\
x_{n}%
\end{array}
\right)  =\left(
\begin{array}
[c]{c}%
b_{1}\\
b_{2}\\
\vdots\\
b_{m}%
\end{array}
\right)  . \label{eq.exam.systems.gen}%
\end{equation}
In other words, it is equivalent to the vector equation%
\begin{equation}
Ax=b, \label{eq.exam.systems.gen.short}%
\end{equation}
where%
\begin{align*}
A  &  =\left(
\begin{array}
[c]{cccc}%
a_{1,1} & a_{1,2} & \cdots & a_{1,n}\\
a_{2,1} & a_{2,2} & \cdots & a_{2,n}\\
\vdots & \vdots & \ddots & \vdots\\
a_{m,1} & a_{m,2} & \cdots & a_{m,n}%
\end{array}
\right)  ,\\
x  &  =\left(
\begin{array}
[c]{c}%
x_{1}\\
x_{2}\\
\vdots\\
x_{n}%
\end{array}
\right)  ,\ \ \ \ \ \ \ \ \ \ b=\left(
\begin{array}
[c]{c}%
b_{1}\\
b_{2}\\
\vdots\\
b_{m}%
\end{array}
\right)  .
\end{align*}
(Some authors write the $A$, the $x$ and the $b$ in
(\ref{eq.exam.systems.gen.short}) in boldface in order to stress that these
are vectors, not numbers. We shall not.) The matrix $A$ and the vector $b$ are
known; the vector $x$ is what we want to find.

Thus, matrices give us a way to rewrite systems of linear equations as single
equations between vectors. Moreover, as we will see, they give us a way to
manipulate these equations easily.

To solve a vector equation like (\ref{eq.exam.systems.gen.short}) means (in
some sense) to \textquotedblleft undo\textquotedblright\ a matrix
multiplication. In fact, if we could divide by a matrix, then we could
immediately solve $Ax=b$ by \textquotedblleft dividing by $A$%
\textquotedblright. Unfortunately, we cannot divide by a matrix in general.
But the idea is fruitful: In fact, some matrices $A$ are invertible (i.e.,
have an inverse $A^{-1}$), and for those matrices, we can transform $Ax=b$
into $x=A^{-1}b$, which gives us an explicit and unique solution for the
system (\ref{eq.exam.systems.gen}). This doesn't work for all $A$ (since not
all $A$ are invertible), and is not a very practical way of solving systems of
linear equations; but the notion of invertible matrices is rather important,
so we begin by studying them.

\subsection{Inverse matrices}

\begin{definition}
\label{def.inverse-matrix}Let $n\in\mathbb{N}$ and $m\in\mathbb{N}$. Let $A$
be an $n\times m$-matrix, and $B$ be an $m\times n$-matrix.

\textbf{(a)} We say that $B$ is \textit{right-inverse} to $A$ if $AB=I_{n}$.

\textbf{(b)} We say that $B$ is \textit{left-inverse} to $A$ if $BA=I_{m}$.

\textbf{(c)} We say that $B$ is \textit{inverse} to $A$ if both $AB=I_{n}$ and
$BA=I_{m}$.
\end{definition}

\begin{example}
\textbf{(a)} Let $A=\left(  1,4\right)  $. (Recall that this means the
$1\times2$-matrix $\left(
\begin{array}
[c]{cc}%
1 & 4
\end{array}
\right)  $.) When is a matrix $B$ right-inverse to $A$ ?

First, if $B$ is right-inverse to $A$, then $B$ must be a $2\times1$-matrix
(since any right-inverse to an $n\times m$-matrix has to be an $m\times
n$-matrix). So let us assume that $B$ is a $2\times1$-matrix. Thus, $B$ must
have the form $B=\left(
\begin{array}
[c]{c}%
u\\
v
\end{array}
\right)  $ for some numbers $u$ and $v$. Then, $AB=\left(
\begin{array}
[c]{cc}%
1 & 4
\end{array}
\right)  \left(
\begin{array}
[c]{c}%
u\\
v
\end{array}
\right)  =\left(
\begin{array}
[c]{c}%
1u+4v
\end{array}
\right)  $. In order for $B$ to be right-inverse to $A$, it is necessary and
sufficient that $AB=I_{1}$ (because this is how we defined \textquotedblleft
right-inverse\textquotedblright). In other words, we must have $\left(
\begin{array}
[c]{c}%
1u+4v
\end{array}
\right)  =\left(
\begin{array}
[c]{c}%
1
\end{array}
\right)  $ (since $AB=\left(
\begin{array}
[c]{c}%
1u+4v
\end{array}
\right)  $ and $I_{1}=\left(
\begin{array}
[c]{c}%
1
\end{array}
\right)  $). In other words, we must have $1u+4v=1$.

Hence, a matrix $B$ is right-inverse to $A$ if and only if it has the form
$B=\left(
\begin{array}
[c]{c}%
u\\
v
\end{array}
\right)  $ for some numbers $u$ and $v$ satisfying $1u+4v=1$. How do we find
two such numbers $u$ and $v$ ? Well, we can view $1u+4v=1$ as a system of $1$
linear equation in $2$ variables, but actually we can just read off the
solution: $v$ can be chosen arbitrarily, and $u$ then has to be $1-4v$. Hence,
a matrix $B$ is right-inverse to $A$ if and only if it has the form $B=\left(
%
\begin{array}
[c]{c}%
1-4v\\
v
\end{array}
\right)  $ for some number $v$. In particular, there are \textbf{infinitely
many} matrices $B$ that are right-inverse to $A$ (because we have full freedom
in choosing $v$).

\textbf{(b)} Let $A=\left(  1,4\right)  $ again. When is a matrix $B$
left-inverse to $A$ ?

Again, $B$ must be a $2\times1$-matrix in order for this to have any chance of
being true. So let us assume that $B$ is a $2\times1$-matrix, and write $B$ in
the form $B=\left(
\begin{array}
[c]{c}%
u\\
v
\end{array}
\right)  $ for some numbers $u$ and $v$. Then, $BA=\left(
\begin{array}
[c]{c}%
u\\
v
\end{array}
\right)  \left(  1,4\right)  =\left(
\begin{array}
[c]{cc}%
u\cdot1 & u\cdot4\\
v\cdot1 & v\cdot4
\end{array}
\right)  $. In order for $B$ to be left-inverse to $A$, it is necessary and
sufficient that $BA=I_{2}$ (because this is how we defined \textquotedblleft
left-inverse\textquotedblright). In other words, we must have $\left(
\begin{array}
[c]{cc}%
u\cdot1 & u\cdot4\\
v\cdot1 & v\cdot4
\end{array}
\right)  =\left(
\begin{array}
[c]{cc}%
1 & 0\\
0 & 1
\end{array}
\right)  $ (since $BA=\left(
\begin{array}
[c]{cc}%
u\cdot1 & u\cdot4\\
v\cdot1 & v\cdot4
\end{array}
\right)  $ and $I_{2}=\left(
\begin{array}
[c]{cc}%
1 & 0\\
0 & 1
\end{array}
\right)  $). In other words, we must have%
\[
\left\{
\begin{array}
[c]{c}%
u\cdot1=1;\\
u\cdot4=0;\\
v\cdot1=0;\\
v\cdot4=1
\end{array}
\right.  .
\]
But this cannot happen! Indeed, the equations $u\cdot1=1$ and $u\cdot4=0$
contradict each other (because if $u\cdot1=1$, then $u\cdot4$ must be $4$).
Hence, $B$ can never be left-inverse to $A$. In other words, the matrix $A$
\textbf{has no} left-inverse.

\textbf{(c)} Let $A=\left(
\begin{array}
[c]{c}%
1\\
4
\end{array}
\right)  $. When is a matrix $B$ left-inverse to $A$ ? When is a matrix $B$
right-inverse to $A$ ? I will let you figure this out (see Exercise
\ref{exe.exam.inverses} below).

\textbf{(d)} Let $A=\left(
\begin{array}
[c]{cc}%
1 & -1\\
1 & 1
\end{array}
\right)  $. When is a matrix $B$ left-inverse to $A$ ? When is a matrix $B$
right-inverse to $A$ ? [...]
\end{example}

\begin{exercise}
\label{exe.exam.inverses}Let $A=\left(
\begin{array}
[c]{c}%
1\\
4
\end{array}
\right)  $. When is a matrix $B$ left-inverse to $A$ ? When is a matrix $B$
right-inverse to $A$ ?
\end{exercise}

[to be continued]

\begin{thebibliography}{999999999}                                                                                        %


\bibitem[deBoor]{deBoor}Carl de Boor, \textit{An empty exercise}.
\url{ftp://ftp.cs.wisc.edu/Approx/empty.pdf} .

\bibitem[Drucker12]{Drucker12}D. Drucker, \textit{Annotated Bibliography of
Linear Algebra Books}, February 2012.\newline\url{http://www.math.wayne.edu/~drucker/linalgrefsW12.pdf}

\bibitem[OlvSha06]{OlvSha06}Peter J. Olver, Chehrzad Shakiban, \textit{Applied
Linear Algebra}, Prentice Hall, 2006.

\bibitem[Heffer16]{Heffer16}Jim Hefferon, \textit{Linear Algebra}, 2016.
(Scroll down to \textquotedblleft The next edition\textquotedblright\ on
\url{http://joshua.smcvt.edu/linearalgebra/} .)

\bibitem[LaNaSc16]{LaNaSc16}Isaiah Lankham, Bruno Nachtergaele, Anne
Schilling, \textit{Linear Algebra As an Introduction to Abstract Mathematics},
2016. (Download from
\url{https://www.math.ucdavis.edu/~anne/linear_algebra/mat67_course_notes.pdf} .)

\bibitem[Treil15]{Treil15}Sergei Treil, \textit{Linear Algebra Done Wrong},
2015. (Download from
\url{https://www.math.brown.edu/~treil/papers/LADW/book.pdf} .)
\end{thebibliography}


\end{document}